\documentclass{pclass}
%\documentclass[a4paper,twoside,11pt]{book}
%\oddsidemargin 0cm
%\evensidemargin 0cm
%\topmargin 1cm
    
\begin{document}
\titulopro{Cifrado Punto a Punto en HTML5 con KHTML}
\tutor{Jose Ramon Portillo}
\departamento{Matematica Aplicada}
\autores{Eduardo Robles Elvira}{}
\dia{05/2010}
\titulacion{Ingenieria Informatica}

\hacerportada

\makeatletter
\renewcommand*\l@section{\@dottedtocline{1}{0em}{2.5em}}
\renewcommand*\l@subsection{\@dottedtocline{2}{1.5em}{3.2em}}
\renewcommand*\l@subsubsection{\@dottedtocline{3}{4.3em}{3.2em}}
\makeatother

\renewcommand{\frontmatter}{\pagenumbering{Roman}}
\frontmatter
        
\cdpchapter{Resumen}

Cada vez está tomando más importancia la web y multitud de aplicaciones que antes eran típicamente de escritorio ahora lo son web. Estas aplicaciones web son usadas tanto por particulares como por empresas o instituciones como universidades.

Muchas de estas aplicaciones web que han pasado de ser aplicaciones de escritorio a ser aplicaciones web son aplicaciones cuyo fin es la comunicación entre personas, debido a la facilidad con la que las páginas web son capaces de cumplir ese objetivo.

Los clientes de correo electrónico o los clientes de mensajería instantánea son claros ejemplos de aplicaciones que antes eran típicamente de escritorio y que ahora en muchos casos son web. Los protocolos involucrados en este tipo de comunicaciones están diseñados con el concepto cliente-servidor, de manera que cualquier comunicación entre dos clientes pasa necesariamente por un servidor. Siendo así, las aplicaciones de escritorio para mantener una conversación segura entre dos cliente suelen implementar cifrado y firmado punto a punto: los clientes de correo suelen soportar PGP para este efecto, y para la mensajería instantánea se suele utilizar OTR.

Sin embargo, en un mundo web en el que la información discurre con una facilidad nunca antes conocida, el único estándar de seguridad de las comunicaciones de extendido uso en las páginas web es SSL (Secure Socket Layer) mediante HTTPS, que permite mantener una conversación segura entre el cliente y el servidor web, pero no permite una comunicación segura, privada y/o autenticada entre dos clientes porque no implementa seguridad punto a punto entre dos cliente. Por tanto el servidor web que permite la comunicación entre ambos siempre podrá conocer y modificar el contenido de las mismas.

HTTPS es un estándar actualmente muy arraigado en Internet que funciona razonablemente bien para el objetivo que fue diseñado. Gracias a HTTPS muchas personas realizan compras por Internet, administran sus cuentas mediante la banca electrónica y realizan trámites gubernamentales online (entre otros usos), sin necesidad de salir del salón de su casa, de forma segura, y sin necesidad de instalarse ningún software adicional, todo desde su navegador web preferido.

El siguiente paso natural en seguridad para que las aplicaciones web puedan equipararse a nivel de características a las aplicaciones de escritorio es introducir el concepto de cifrado punto a punto (Point-to-point Encryption) directamente en los navegadores web, de forma que sea tan fácil y transparente a nivel de programación web, así como igual de confiable a nivel de usuario de una navegador web, como HTTPS. Ese es el objetivo de este proyecto.

Llevar el concepto de cifrado punto a punto a las páginas web no es una idea nueva, y ya hay empresas ofreciendo servicios de este tipo. En esta memoria también estudiaremos las soluciones ya existentes y razonaremos porqué no son suficientes.


\cdpchapter{Agradecimientos}
Quiero agradecer especialmente a mis amigos su apoyo, puesto que con nuestras conversaciones acerca del cifrado punto a punto y la seguridad he podido averiguar cual era el camino que creo correcto a la hora de realizar este proyecto. También quiero agradecer a mi familia su apoyo y comprensión, porque ellos han sabido comprender mi proceder con el proyecto y consiguieron que nunca tirase la toalla.

Otro bastión de apoyo crucial para la realización de este proyecto han sido mis compañeros de trabajo, gracias a los cuales surgieron buenas ideas y me impulsaron entre otras cosas a dar una charla sobre este proyecto en las Oficinas de Google España. Por último, agradecer a la comunidad de KDE su contribución al software libre mediante el motor de renderizado de páginas web KHTML en el que se basa este proyecto, y a KDE España y la organización del Akademy-es por permitir que diese una ponencia sobre el mismo.


\tableofcontents %�ndice de contenidos
\listoftables %�ndice de cuadros
\listoffigures %�ndice de figuras
\lstlistoflistings %�ndice de c�digos

\mainmatter

% 1. Objetivos: objetivos del proyecto, cifrado punto a punto en khtml en html extensi�n est�ndar
% 1. Objetivos: objetivos del proyecto, cifrado punto a punto en khtml en html extensión estándar
\chapter{Motivación y Objetivos}\label{defobjetivos}
\section{Motivación}\label{motivacion}

Relacionados con la informática, existen dos grandes hitos tecnológicos relevantes: el primero es sin duda el boom tecnológico del microchip gracias al cual hemos conseguido ordenadores de una potencia de cálculo descomunal a bajo coste y nunca antes si quiera soñada. El segundo, es sin duda Internet, que permite comunicarnos entre personas de forma instantánea desde cualquier parte del planeta.

Si unimos ambos hitos tecnológicos, que son la casi infinita capacidad de cálculo de los ordenadores actuales junto con la facilidad de coste marginal de comunicación que nos ofrece la red de redes, obtenemos varios resultados interesantes:

\begin{enumerate}
 \item La comunicación entre personas se realiza cada vez más mediante medios digitales, a través de la red de redes debido al bajo coste y lo fácil que esto resulta, frente a los antiguos métodos de comunicación como pueden ser: teléfono, fax, carta, teletipo, o incluso presencial. Ahora en vez de hablar en persona, hay muchas personas que prefieren hacerlo de forma online.
 \item La red almacena cada vez más información sobre nosotros ``en la nube'', puesto que la capacidad de almacenamiento de los datos es notablemente alta y puesto que almacenar información online permite acceder a ella desde cualquier lugar en cualquier momento.
 \item Gran parte de la información es fácilmente procesable por medios automáticos debido a la gran capacidad de cálculo de los ordenadores actuales.
 \item Los usuarios se acostumbran a servicios que son de coste marginal y por ende a servicios gratuitos cuyo modelo de negocio se basa en la publicidad contextual.
 \item Los gobiernos y empresas se aprovechan de toda esta información que anteriormente era imposible de recopilar, estructurar o usar de forma masiva, y ahora sí que es posible hacerlo de forma masiva y transparente.
\end{enumerate}

No es ningún secreto que empresas como Google basan su modelo de negocio en la información, y tampoco es ningún secreto (de hecho por definición una ley debe ser pública) que los Gobiernos se aprovechan de la digitalización de la información promulgando leyes como la Directiva Europea de Retención de Datos y la transposición de dicha directiva en España, la Ley de Retención de Datos, que especifica que por ejemplo debe guardarse la siguiente información sobre todas las páginas web que visitemos: URL, fecha de visita, dirección IP desde la cual se visitó. Por último y no menos importante, debido a que la información es poder, tampoco hay que descartar la posibilidad de que malhechores intenten beneficiarse de la inseguridad de nuestros datos para hacer negocios ilegales con ellos.

En un mundo cada vez más informatizado, hay una cantidad creciente y preocupante de información sobre nosotros en la red, que inevitablemente pasa por múltiples intermediarios y que no controlamos. Por supuesto, existen medios para conseguir controlar en mayor o menor medida esta información, siendo el cifrado la principal herramienta.

El cifrado consiste en el intercambio de mensajes de manera que sólo puedan ser leídos en claro por personas a las que van dirigidos y que poseen los medios para descifrarlos. Muchos protocolos de comunicación en Internet soportan cifrado, y sería del todo aconsejable que se cifrasen las comunicaciones siempre que fuese posible, no obstante no es así. Sin ir más lejos, el protocolo fundamental bajo el que funcionan las páginas web, HTTP, soporta cifrado mediante su variante HTTPS, pero un muy bajo porcentaje de páginas web lo soporta, y un porcentaje mucho menos lo utiliza por defecto.

Cuando entramos en una página web mediante HTTPS mantenemos una conversación segura con el servidor, de tal manera que la llamada comunicación cliente-servidor no resulta comprometida. Este sistema es bastante robusto y suficente como para que millones de personas realicen sus compras por Internet, se autentiquen de forma segura o administren sus cuentas mediante la banca electrónica.

No obstante cuando nos comunicamos por Internet con otras personas lo corriente no es intentar comunicarnos con personas que se encuentren en el servidor, sino con personas que se encuentran al mismo nivel que nosotros, conectadas al servidor, de tal manera que la conexión segura se establece entre nosotros y el servidor, y luego entre el servidor y la persona con la que queramos hablar, en el mejor de los casos. Los tres puntos que pueden ver el texto de la comunicación en claro son: nosotros, el servidor, y la persona con la que nos comunicamos. Sin embargo, el servidor no es el destinatario de la información y por tanto para mayor seguridad no debería de conocer el contenido de la misma.

HTTPS utiliza para establecer una comunicación segura el protocolo SSL (Secure Socket Layer). Servicios no basados en la web como el correo electrónico (con IMAP o POP) soportan el mismo método de seguridad de las comunicaciones, pero tienen el mismo problema: la comunicación no es segura de ``punto a punto'' sino de ``cliente a servidor''. Lo mismo ocurre con otros servicios de comunicación entre personas (que son los ``puntos`` que se comunican), como la mensajería instantánea o el VoIP: muchos soportan únicamente cifrado de cliente a servidor de forma nativa. 

Para resolver dicho problema, existen diferentes soluciones que implementan la seguridad punto a punto (''end-to-end security``):
\begin{itemize}
 \item Para el correo electrónico se puede utilizar PGP (Pretty Good Privacy) o su variante libre GnuPG para cifrar y firmar los mensajes. Este es un estándar utilizado e implementado por la mayoría de clientes de correo electrónico: KMail, Outlook, Thunderbird, etc.
 \item Para la mensajería instantánea se puede utilizar OTR (Off The Record), igualmente para firmar y cifrar los mensajes. OTR se ha convertido también en un estándar soportado por una plétora de clientes de mensajería instantánea, entre ellos Kopete, Pidgin, o Adium.
 \item Para el VoIP existe otro sistema criptográfico llamado ZRTP y que es una creación de Phil Zimmermann, uno de los creadores del anteriormente mencionado PGP, y que permite ser usado en la mayoría de clientes VoIP.
\end{itemize}

(Es normal que el lector no conozca algunos o todos los términos aquí empleados. Por ello en el siguiente apartado explicaremos muchos de los conceptos que subyacen en el problema que abordamos, y definiremos muchos de los términos técnicos antes mencionados.)

Uno de los principales inconvenientes que surge a la hora de utilizar un sistema seguro de comunicación es que suele ser necesario bien instalarse un software concreto para que este funcione, o bien configurarlo de forma especial. Plantearse usar cifrado o no ya es una barrera infranqueable para la mayoría de los usuarios de ordenadores puesto que no están concienciados en el problema que supone la pérdida de privacidad inherente a enviar mensajes en claro por la red, porque ni se dan cuenta de que ésto está ocurriendo ni de las repercusiones a largo o medio plazo que ésto conlleva.

Por otra parte, podría pensarse que al menos la tecnología está ahí y sólo es necesario llevar a cabo un movimiento de concienciación social en el tema de la seguridad para que empiece a usarse. En parte esto es cierto, y por ejemplo hay software como Adium (uno de los clientes de mensajería instantánea antes mencionado) que viene configurado por defecto con OTR activado de manera que utiliza lo que es conocido como ''cifrado oportunista``: si el usuario comienza una conversación con una persona que esté utilizando también un cliente de mensajería instantánea con soporte de OTR, Adium se da cuenta y automáticamente comienza una sesión de OTR segura en la que los mensajes van firmados y cifrados.

No obstante incluso si los clientes de comunicaciones de escritorio parece que cada vez son más amigables en este sentido, también debemos contar con el factor web. Y es que cada vez más aplicaciones que antes eran típicamente aplicaciones de escritorio, ahora lo son directamente web. Ésto está ocurriendo con Gmail por ejemplo, que ahora sirve tanto como cliente de correo como de mensajería instantánea. Y recientemente ahora éstas aplicaciones web cada vez soportan más características, como por ejemplo videoconferencia y conversaciones de voz. 

Es decir, mediante los clientes web tenemos la mayoría de las características que se encontraban en los clientes de escritorio, pero con la ventaja de que a diferencia de éstos, no hace falta instalarse ni configurar ningún software (a parte del navegador) para hacerlos funcionar, y están disponibles en cualquier navegador web y sistema operativo, desde dispositivos móviles a ordenadores de sobremesa.

Si bien con los clientes de escritorio era posible mantener cierto grado de seguridad ''end-to-end``, pese a no haber llegado a un uso generalizado de estos sistemas de cifrado de las comunicaciones, en los sistemas de comunicación via web nos encontramos con una situación desoladora, puesto que la seguridad punto a punto no se implementa de forma nativa en prácticamente ningún de estos servicios web y de hecho, como veremos a lo largo de este documento, esto es debido entre otras cosas a que no existe ninguna manera confiable y sencilla de implementarla. 

Mientras que las páginas web cada vez se utilizan más para comunicar a las personas, la privacidad online se deteriora. Tal y como mencinaban en un artículo recientemente en slashdot \cite{onlineprivacybroken}, la industria del software no focaliza sus esfuerzos en mantener la privacidad de sus clientes porque no es lo que sus clientes no se lo piden. Y cuando  se lo piden, entonces es una ocurriencia tardía. La seguridad y privacidad son conceptos que no son fácilmente empotrables en un sistema si no ha sido diseñado para ello.

Todo esto es lo que motivó al autor a realizar su proyecto de fin de carrera sobre un tema tan controvertido. En la siguiente sección se detallarán los objetivos que se plantean en base a los motivos anteriormente expuestos.

\section{Objetivos}\label{objetivos}


% 2. Introducci�n, conceptos. arquitectura cliente-servidor, man-in-the-middle, server-in-the-middle, https
% 2. Introducción, conceptos. arquitectura cliente-servidor, man-in-the-middle, server-in-the-middle, https
\chapter{Introducción}\label{introduccion}
\section{Definiciones y abreviaturas}\label{definicionesyabreviaturas}

Esto es un cap\'itulo de prueba. Esto es un cap\'itulo de prueba.Esto es un cap\'itulo de prueba.
Esto es un cap\'itulo de prueba. Esto es un cap\'itulo de prueba. Esto es un cap\'itulo de prueba. 
Esto es un cap\'itulo de prueba. Esto es un cap\'itulo de prueba. Esto es un cap\'itulo de prueba. 
Esto es un cap\'itulo de prueba. 

Esto es un cap\'itulo de prueba. Esto es un cap\'itulo de prueba. Esto es un cap\'itulo de prueba. 
Esto es un cap\'itulo de prueba. Esto es un cap\'itulo de prueba. Esto es un cap\'itulo de prueba. 
Esto es un cap\'itulo de prueba. 

Esto es una referncia a la bibliograf\'ia \cite{desousa}.



\codigofuente{TeX}{Macro para insertar un cuadro}{codigo/macrotabla}

\figura{0.5}{img/knuth}{Im\'agen de Knuth}{knuth}{}

% 3. An�lisis de antecedentes y aportaci�n realizada
% 3.1 Exposici�n del problema
% 3.1.1 Antecedentes Aplicaciones de Escritorio (clientes de correo y mensajer�a instant�nea con PGP/OTR)
% 3.1.2 Antecedentes Aplicaciones web (actualidad, HTTPS)
% 3.1.3 Hushmail (objetivo y limitaciones)
% 3.1.4 XML Signature (objetivo y limitaciones)
% 3.2 Soluci�n propuesta
% 3.2.1 Posibles soluciones barajadas
% 3.2.2 Soluci�n elegida
% 3. Análisis de antecedentes y aportación realizada
% 3.1 Exposición del problema
% 3.1.1 Antecedentes Aplicaciones de Escritorio (clientes de correo y mensajería instantánea con PGP/OTR)
% 3.1.2 Antecedentes Aplicaciones web (actualidad, HTTPS)
% 3.1.3 Hushmail (objetivo y limitaciones)
% 3.1.4 XML Signature (objetivo y limitaciones)
% 3.2 Solución propuesta
% 3.2.1 Posibles soluciones barajadas
% 3.2.2 Solución elegida
\chapter{Análisis de antecedentes y aportación realizada}\label{antecedentes}
\section{Exposición del problema}\label{exposicionproblema}
\subsection{Aplicaciones de Escritorio}\label{aplicacionesdeescritorio}
\subsection{Aplicaciones web}\label{aplicacionesweb}
\subsection{Hushmail}\label{hushmail}
\subsection{XML Signature}\label{xmlsignature}
\section{Solución propuesta}\label{solucionpropuesta}
\subsection{Posibles soluciones barajadas}\label{posiblessoluciones}
\subsection{Solución elegida}\label{solucionelegida}
% 4 Requisitos: qu� buscamos que tenga nuestra extensi�n e implementaci�n
\chapter{Análisis de requisitos}\label{requisitos}

En este capítulo describiremos cuales son los requisitos específicos que buscamos que vaya a cumplir nuestro proyecto. El objetivo a nivel de programación es conseguir extender HTML de la forma más sencilla posible para soportar seguridad de punto a punto, implementar dicha extensión en el motor de renderizado KHTML de KDE. También será necesario crear un página web de ejemplo que utilice las características que brinda esta nueva extensión de manera que se pueda ver fácilmente cual es el posible uso útil de dicha extensión.

La manera específica en que se consigue implementar este sistema de seguridad no será abordada en este capítulo sino en el siguiente, en el que se explica las soluciones que hay para abordar el problema, para luego explicar aquella por la cual optamos y las razones que respaldan esa decisión. También es relevante mencionar aquí el capítulo que le sigue, que es el de Implementación, donde se detalla el funcionamiento interno de KHTML para luego continuar explicando cómo se ha modificado KHTML para conseguir implementar la extensión propuesta, y qué medidas de seguridad se han adoptado.

Esta extensión de HTML debe permitir a los desarrolladores de sitios web utilizar seguridad de punto a punto sin que los usuarios tengan que confiar en que el sitio web utilice el cifrado de forma correcta, sino que por el contrario el usuario sólo deba confiar en su navegador web. El cifrado y descifrado debe realizarse por el navegador y la página web no debe nunca poder acceder al texto en claro. El internauta que navegue por páginas web que permitan la comunicación segura mediante este nuevo sistema debe poder de alguna manera estandarizada reconocer cuando un mensaje está siendo enviado de manera segura y conocer en todo momento a quien se está enviando el mensaje. Igualmente, debe poder reconocer visualmente cuándo un mensaje ha sido recibido de forma segura, y comprobar los datos del mensaje.

Debido a que esta extensión tiene como objetivo aumentar la seguridad de los Interanutas a la hora de utilizar Internet como una herramienta de comunicación, uno de los requisitos que debemos cumplir es implementar las medidas de seguridad necesarias para que un sitio web malicioso no pueda ni acceder al texto en claro saltándose las medidas de seguridad que implemente el navegador, ni engañar al usuario haciéndole creer que el mensaje que esté recibiendo o enviando lo esté haciendo de forma segura.

% 5 An�lisis temporal del desarrollo
\chapter{An\'alisis temporal y costes de desarrollo}\label{anatemporal}

\section{An\'alisis temporal}
	
\section{Costes de desarrollo}		
	

	
% 6. Pruebas y conclusiones
\chapter{Pruebas y conclusiones}\label{pruebas}

\section{Pruebas}

	
\section{Conclusiones}\label{conclusiones} 


% 7. Manuales para empaquetadores, desarrolladores y usuarios
\chapter{Manual}\label{manual}



\backmatter

\chapter{Apéndices}\label{apendices}
 
\chapter{Licencia}\label{licencia}
\begin{center}
                    GNU GENERAL PUBLIC LICENSE\\
                       Version 2, June 1991\\*[2ex]

 Copyright (C) 1989, 1991 Free Software Foundation, Inc.\\
     59 Temple Place, Suite 330, Boston, MA  02111-1307  USA\\
 Everyone is permitted to copy and distribute verbatim copies\\
 of this license document, but changing it is not allowed.
\end{center}

\begin{center}
Preamble
\end{center}

  The licenses for most software are designed to take away your
freedom to share and change it.  By contrast, the GNU General Public
License is intended to guarantee your freedom to share and change free
software--to make sure the software is free for all its users.  This
General Public License applies to most of the Free Software
Foundation's software and to any other program whose authors commit to
using it.  (Some other Free Software Foundation software is covered by
the GNU Library General Public License instead.)  You can apply it to
your programs, too.

  When we speak of free software, we are referring to freedom, not
price.  Our General Public Licenses are designed to make sure that you
have the freedom to distribute copies of free software (and charge for
this service if you wish), that you receive source code or can get it
if you want it, that you can change the software or use pieces of it
in new free programs; and that you know you can do these things.

  To protect your rights, we need to make restrictions that forbid
anyone to deny you these rights or to ask you to surrender the rights.
These restrictions translate to certain responsibilities for you if you
distribute copies of the software, or if you modify it.

  For example, if you distribute copies of such a program, whether
gratis or for a fee, you must give the recipients all the rights that
you have.  You must make sure that they, too, receive or can get the
source code.  And you must show them these terms so they know their
rights.

  We protect your rights with two steps: (1) copyright the software, and
(2) offer you this license which gives you legal permission to copy,
distribute and/or modify the software.

  Also, for each author's protection and ours, we want to make certain
that everyone understands that there is no warranty for this free
software.  If the software is modified by someone else and passed on, we
want its recipients to know that what they have is not the original, so
that any problems introduced by others will not reflect on the original
authors' reputations.

  Finally, any free program is threatened constantly by software
patents.  We wish to avoid the danger that redistributors of a free
program will individually obtain patent licenses, in effect making the
program proprietary.  To prevent this, we have made it clear that any
patent must be licensed for everyone's free use or not licensed at all.

  The precise terms and conditions for copying, distribution and
modification follow.

\begin{center}
            GNU GENERAL PUBLIC LICENSE\\
   TERMS AND CONDITIONS FOR COPYING, DISTRIBUTION AND MODIFICATION
\end{center}

  0. This License applies to any program or other work which contains
a notice placed by the copyright holder saying it may be distributed
under the terms of this General Public License.  The "Program", below,
refers to any such program or work, and a "work based on the Program"
means either the Program or any derivative work under copyright law:
that is to say, a work containing the Program or a portion of it,
either verbatim or with modifications and/or translated into another
language.  (Hereinafter, translation is included without limitation in
the term "modification".)  Each licensee is addressed as "you".

\noindent
Activities other than copying, distribution and modification are not
covered by this License; they are outside its scope.  The act of
running the Program is not restricted, and the output from the Program
is covered only if its contents constitute a work based on the
Program (independent of having been made by running the Program).
Whether that is true depends on what the Program does.

  1. You may copy and distribute verbatim copies of the Program's
source code as you receive it, in any medium, provided that you
conspicuously and appropriately publish on each copy an appropriate
copyright notice and disclaimer of warranty; keep intact all the
notices that refer to this License and to the absence of any warranty;
and give any other recipients of the Program a copy of this License
along with the Program.

\noindent
You may charge a fee for the physical act of transferring a copy, and
you may at your option offer warranty protection in exchange for a fee.

  2. You may modify your copy or copies of the Program or any portion
of it, thus forming a work based on the Program, and copy and
distribute such modifications or work under the terms of Section 1
above, provided that you also meet all of these conditions:

    a) You must cause the modified files to carry prominent notices
    stating that you changed the files and the date of any change.

    b) You must cause any work that you distribute or publish, that in
    whole or in part contains or is derived from the Program or any
    part thereof, to be licensed as a whole at no charge to all third
    parties under the terms of this License.

    c) If the modified program normally reads commands interactively
    when run, you must cause it, when started running for such
    interactive use in the most ordinary way, to print or display an
    announcement including an appropriate copyright notice and a
    notice that there is no warranty (or else, saying that you provide
    a warranty) and that users may redistribute the program under
    these conditions, and telling the user how to view a copy of this
    License.  (Exception: if the Program itself is interactive but
    does not normally print such an announcement, your work based on
    the Program is not required to print an announcement.)

\noindent
These requirements apply to the modified work as a whole.  If
identifiable sections of that work are not derived from the Program,
and can be reasonably considered independent and separate works in
themselves, then this License, and its terms, do not apply to those
sections when you distribute them as separate works.  But when you
distribute the same sections as part of a whole which is a work based
on the Program, the distribution of the whole must be on the terms of
this License, whose permissions for other licensees extend to the
entire whole, and thus to each and every part regardless of who wrote it.

\noindent
Thus, it is not the intent of this section to claim rights or contest
your rights to work written entirely by you; rather, the intent is to
exercise the right to control the distribution of derivative or
collective works based on the Program.

\noindent
In addition, mere aggregation of another work not based on the Program
with the Program (or with a work based on the Program) on a volume of
a storage or distribution medium does not bring the other work under
the scope of this License.

  3. You may copy and distribute the Program (or a work based on it,
under Section 2) in object code or executable form under the terms of
Sections 1 and 2 above provided that you also do one of the following:
\begin{quote}
    a) Accompany it with the complete corresponding machine-readable
    source code, which must be distributed under the terms of Sections
    1 and 2 above on a medium customarily used for software interchange; or,

    b) Accompany it with a written offer, valid for at least three
    years, to give any third party, for a charge no more than your
    cost of physically performing source distribution, a complete
    machine-readable copy of the corresponding source code, to be
    distributed under the terms of Sections 1 and 2 above on a medium
    customarily used for software interchange; or,

    c) Accompany it with the information you received as to the offer
    to distribute corresponding source code.  (This alternative is
    allowed only for noncommercial distribution and only if you
    received the program in object code or executable form with such
    an offer, in accord with Subsection b above.)
\end{quote}
\noindent
The source code for a work means the preferred form of the work for
making modifications to it.  For an executable work, complete source
code means all the source code for all modules it contains, plus any
associated interface definition files, plus the scripts used to
control compilation and installation of the executable.  However, as a
special exception, the source code distributed need not include
anything that is normally distributed (in either source or binary
form) with the major components (compiler, kernel, and so on) of the
operating system on which the executable runs, unless that component
itself accompanies the executable.

\noindent
If distribution of executable or object code is made by offering
access to copy from a designated place, then offering equivalent
access to copy the source code from the same place counts as
distribution of the source code, even though third parties are not
compelled to copy the source along with the object code.

  4. You may not copy, modify, sublicense, or distribute the Program
except as expressly provided under this License.  Any attempt
otherwise to copy, modify, sublicense or distribute the Program is
void, and will automatically terminate your rights under this License.
However, parties who have received copies, or rights, from you under
this License will not have their licenses terminated so long as such
parties remain in full compliance.

  5. You are not required to accept this License, since you have not
signed it.  However, nothing else grants you permission to modify or
distribute the Program or its derivative works.  These actions are
prohibited by law if you do not accept this License.  Therefore, by
modifying or distributing the Program (or any work based on the
Program), you indicate your acceptance of this License to do so, and
all its terms and conditions for copying, distributing or modifying
the Program or works based on it.

  6. Each time you redistribute the Program (or any work based on the
Program), the recipient automatically receives a license from the
original licensor to copy, distribute or modify the Program subject to
these terms and conditions.  You may not impose any further
restrictions on the recipients' exercise of the rights granted herein.
You are not responsible for enforcing compliance by third parties to
this License.

  7. If, as a consequence of a court judgment or allegation of patent
infringement or for any other reason (not limited to patent issues),
conditions are imposed on you (whether by court order, agreement or
otherwise) that contradict the conditions of this License, they do not
excuse you from the conditions of this License.  If you cannot
distribute so as to satisfy simultaneously your obligations under this
License and any other pertinent obligations, then as a consequence you
may not distribute the Program at all.  For example, if a patent
license would not permit royalty-free redistribution of the Program by
all those who receive copies directly or indirectly through you, then
the only way you could satisfy both it and this License would be to
refrain entirely from distribution of the Program.

\noindent
If any portion of this section is held invalid or unenforceable under
any particular circumstance, the balance of the section is intended to
apply and the section as a whole is intended to apply in other
circumstances.

\noindent
It is not the purpose of this section to induce you to infringe any
patents or other property right claims or to contest validity of any
such claims; this section has the sole purpose of protecting the
integrity of the free software distribution system, which is
implemented by public license practices.  Many people have made
generous contributions to the wide range of software distributed
through that system in reliance on consistent application of that
system; it is up to the author/donor to decide if he or she is willing
to distribute software through any other system and a licensee cannot
impose that choice.

\noindent
This section is intended to make thoroughly clear what is believed to
be a consequence of the rest of this License.

  8. If the distribution and/or use of the Program is restricted in
certain countries either by patents or by copyrighted interfaces, the
original copyright holder who places the Program under this License
may add an explicit geographical distribution limitation excluding
those countries, so that distribution is permitted only in or among
countries not thus excluded.  In such case, this License incorporates
the limitation as if written in the body of this License.

  9. The Free Software Foundation may publish revised and/or new versions
of the General Public License from time to time.  Such new versions will
be similar in spirit to the present version, but may differ in detail to
address new problems or concerns.

Each version is given a distinguishing version number.  If the Program
specifies a version number of this License which applies to it and "any
later version", you have the option of following the terms and conditions
either of that version or of any later version published by the Free
Software Foundation.  If the Program does not specify a version number of
this License, you may choose any version ever published by the Free Software
Foundation.

  10. If you wish to incorporate parts of the Program into other free
programs whose distribution conditions are different, write to the author
to ask for permission.  For software which is copyrighted by the Free
Software Foundation, write to the Free Software Foundation; we sometimes
make exceptions for this.  Our decision will be guided by the two goals
of preserving the free status of all derivatives of our free software and
of promoting the sharing and reuse of software generally.

\begin{center}
                NO WARRANTY
\end{center}

  11. BECAUSE THE PROGRAM IS LICENSED FREE OF CHARGE, THERE IS NO WARRANTY
FOR THE PROGRAM, TO THE EXTENT PERMITTED BY APPLICABLE LAW.  EXCEPT WHEN
OTHERWISE STATED IN WRITING THE COPYRIGHT HOLDERS AND/OR OTHER PARTIES
PROVIDE THE PROGRAM "AS IS" WITHOUT WARRANTY OF ANY KIND, EITHER EXPRESSED
OR IMPLIED, INCLUDING, BUT NOT LIMITED TO, THE IMPLIED WARRANTIES OF
MERCHANTABILITY AND FITNESS FOR A PARTICULAR PURPOSE.  THE ENTIRE RISK AS
TO THE QUALITY AND PERFORMANCE OF THE PROGRAM IS WITH YOU.  SHOULD THE
PROGRAM PROVE DEFECTIVE, YOU ASSUME THE COST OF ALL NECESSARY SERVICING,
REPAIR OR CORRECTION.

  12. IN NO EVENT UNLESS REQUIRED BY APPLICABLE LAW OR AGREED TO IN WRITING
WILL ANY COPYRIGHT HOLDER, OR ANY OTHER PARTY WHO MAY MODIFY AND/OR
REDISTRIBUTE THE PROGRAM AS PERMITTED ABOVE, BE LIABLE TO YOU FOR DAMAGES,
INCLUDING ANY GENERAL, SPECIAL, INCIDENTAL OR CONSEQUENTIAL DAMAGES ARISING
OUT OF THE USE OR INABILITY TO USE THE PROGRAM (INCLUDING BUT NOT LIMITED
TO LOSS OF DATA OR DATA BEING RENDERED INACCURATE OR LOSSES SUSTAINED BY
YOU OR THIRD PARTIES OR A FAILURE OF THE PROGRAM TO OPERATE WITH ANY OTHER
PROGRAMS), EVEN IF SUCH HOLDER OR OTHER PARTY HAS BEEN ADVISED OF THE
POSSIBILITY OF SUCH DAMAGES.

\begin{center}
             END OF TERMS AND CONDITIONS

        How to Apply These Terms to Your New Programs
\end{center}

  If you develop a new program, and you want it to be of the greatest
possible use to the public, the best way to achieve this is to make it
free software which everyone can redistribute and change under these terms.

  To do so, attach the following notices to the program.  It is safest
to attach them to the start of each source file to most effectively
convey the exclusion of warranty; and each file should have at least
the "copyright" line and a pointer to where the full notice is found.

\begin{quote}
    $<$one line to give the program's name and a brief idea of what it does.$>$\\
    Copyright (C) $<$year$>$  $<$name of author$>$

    This program is free software; you can redistribute it and/or modify
    it under the terms of the GNU General Public License as published by
    the Free Software Foundation; either version 2 of the License, or
    (at your option) any later version.

    This program is distributed in the hope that it will be useful,
    but WITHOUT ANY WARRANTY; without even the implied warranty of
    MERCHANTABILITY or FITNESS FOR A PARTICULAR PURPOSE.  See the
    GNU General Public License for more details.

    You should have received a copy of the GNU General Public License
    along with this program; if not, write to the Free Software
    Foundation, Inc., 59 Temple Place, Suite 330, Boston, MA  02111-1307  USA
\end{quote}

\noindent
Also add information on how to contact you by electronic and paper mail.

\noindent
If the program is interactive, make it output a short notice like this
when it starts in an interactive mode:

\begin{quote}
    Gnomovision version 69, Copyright (C) year  name of author
    Gnomovision comes with ABSOLUTELY NO WARRANTY; for details type `show w'.
    This is free software, and you are welcome to redistribute it
    under certain conditions; type `show c' for details.
\end{quote}

\noindent
The hypothetical commands `show w' and `show c' should show the appropriate
parts of the General Public License.  Of course, the commands you use may
be called something other than `show w' and `show c'; they could even be
mouse-clicks or menu items--whatever suits your program.

\noindent
You should also get your employer (if you work as a programmer) or your
school, if any, to sign a ``copyright disclaimer'' for the program, if
necessary.  Here is a sample; alter the names:

\begin{quote}
  Yoyodyne, Inc., hereby disclaims all copyright interest in the program\\
  `Gnomovision' (which makes passes at compilers) written by James Hacker.\\*[2ex]

  $<$signature of Ty Coon$>$, 1 April 1989\\
  Ty Coon, President of Vice
\end{quote}

\noindent
This General Public License does not permit incorporating your program into
proprietary programs.  If your program is a subroutine library, you may
consider it more useful to permit linking proprietary applications with the
library.  If this is what you want to do, use the GNU Library General
Public License instead of this License.

\bibliographystyle{pfc}
\bibliography{pfcbib}

\end{document}
