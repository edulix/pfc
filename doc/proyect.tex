\documentclass{pclass}
%\documentclass[a4paper,twoside,11pt]{book}
%\oddsidemargin 0cm
%\evensidemargin 0cm
%\topmargin 1cm
    
\begin{document}
\titulopro{Cifrado Punto a Punto en HTML5 con KHTML}
\tutor{Jose Ramon Portillo}
\departamento{Matematica Aplicada}
\autores{Eduardo Robles Elvira}{}
\dia{05/2010}
\titulacion{Ingenieria Informatica}

\hacerportada

\input{tocdef}
\frontmatter
        
\cdpchapter{Resumen}

Cada vez está tomando más importancia la web y multitud de aplicaciones que antes eran típicamente de escritorio ahora lo son web. Estas aplicaciones web son usadas tanto por particulares como por empresas o instituciones como universidades.

Muchas de estas aplicaciones web que han pasado de ser aplicaciones de escritorio a ser aplicaciones web son aplicaciones cuyo fin es la comunicación entre personas, debido a la facilidad con la que las páginas web son capaces de cumplir ese objetivo.

Los clientes de correo electrónico o los clientes de mensajería instantánea son claros ejemplos de aplicaciones que antes eran típicamente de escritorio y que ahora en muchos casos son web. Los protocolos involucrados en este tipo de comunicaciones están diseñados con el concepto cliente-servidor, de manera que cualquier comunicación entre dos clientes pasa necesariamente por un servidor. Siendo así, las aplicaciones de escritorio para mantener una conversación segura entre dos cliente suelen implementar cifrado y firmado punto a punto: los clientes de correo suelen soportar PGP para este efecto, y para la mensajería instantánea se suele utilizar OTR.

Sin embargo, en un mundo web en el que la información discurre con una facilidad nunca antes conocida, el único estándar de seguridad de las comunicaciones de extendido uso en las páginas web es SSL (Secure Socket Layer) mediante HTTPS, que permite mantener una conversación segura entre el cliente y el servidor web, pero no permite una comunicación segura, privada y/o autenticada entre dos clientes porque no implementa seguridad punto a punto entre dos cliente. Por tanto el servidor web que permite la comunicación entre ambos siempre podrá conocer y modificar el contenido de las mismas.

HTTPS es un estándar actualmente muy arraigado en Internet que funciona razonablemente bien para el objetivo que fue diseñado. Gracias a HTTPS muchas personas realizan compras por Internet, administran sus cuentas mediante la banca electrónica y realizan trámites gubernamentales online (entre otros usos), sin necesidad de salir del salón de su casa, de forma segura, y sin necesidad de instalarse ningún software adicional, todo desde su navegador web preferido.

El siguiente paso natural en seguridad para que las aplicaciones web puedan equipararse a nivel de características a las aplicaciones de escritorio es introducir el concepto de cifrado punto a punto (Point-to-point Encryption) directamente en los navegadores web, de forma que sea tan fácil y transparente a nivel de programación web, así como igual de confiable a nivel de usuario de una navegador web, como HTTPS. Ese es el objetivo de este proyecto.

Llevar el concepto de cifrado punto a punto a las páginas web no es una idea nueva, y ya hay empresas ofreciendo servicios de este tipo. En esta memoria también estudiaremos las soluciones ya existentes y razonaremos porqué no son suficientes.


\cdpchapter{Agradecimientos}
Quiero agradecer especialmente a mis amigos su apoyo, puesto que con nuestras conversaciones acerca del cifrado punto a punto y la seguridad he podido averiguar cual era el camino que creo correcto a la hora de realizar este proyecto. También quiero agradecer a mi familia su apoyo y comprensión, porque ellos han sabido comprender mi proceder con el proyecto y consiguieron que nunca tirase la toalla.

Otro bastión de apoyo crucial para la realización de este proyecto han sido mis compañeros de trabajo, gracias a los cuales surgieron buenas ideas y me impulsaron entre otras cosas a dar una charla sobre este proyecto en las Oficinas de Google España. Por último, agradecer a la comunidad de KDE su contribución al software libre mediante el motor de renderizado de páginas web KHTML en el que se basa este proyecto, y a KDE España y la organización del Akademy-es por permitir que diese una ponencia sobre el mismo.


\tableofcontents %�ndice de contenidos
\listoftables %�ndice de cuadros
\listoffigures %�ndice de figuras
\lstlistoflistings %�ndice de c�digos

\mainmatter

% 1. Objetivos: objetivos del proyecto, cifrado punto a punto en khtml en html extensi�n est�ndar
% 1. Objetivos: objetivos del proyecto, cifrado punto a punto en khtml en html extensión estándar
\chapter{Motivación y Objetivos}\label{defobjetivos}
\section{Motivación}\label{motivacion}

Relacionados con la informática, existen dos grandes hitos tecnológicos relevantes: el primero es sin duda el boom tecnológico del microchip gracias al cual hemos conseguido ordenadores de una potencia de cálculo descomunal a bajo coste y nunca antes si quiera soñada. El segundo, es sin duda Internet, que permite comunicarnos entre personas de forma instantánea desde cualquier parte del planeta.

Si unimos ambos hitos tecnológicos, que son la casi infinita capacidad de cálculo de los ordenadores actuales junto con la facilidad de coste marginal de comunicación que nos ofrece la red de redes, obtenemos varios resultados interesantes:

\begin{enumerate}
 \item La comunicación entre personas se realiza cada vez más mediante medios digitales, a través de la red de redes debido al bajo coste y lo fácil que esto resulta, frente a los antiguos métodos de comunicación como pueden ser: teléfono, fax, carta, teletipo, o incluso presencial. Ahora en vez de hablar en persona, hay muchas personas que prefieren hacerlo de forma online.
 \item La red almacena cada vez más información sobre nosotros ``en la nube'', puesto que la capacidad de almacenamiento de los datos es notablemente alta y puesto que almacenar información online permite acceder a ella desde cualquier lugar en cualquier momento.
 \item Gran parte de la información es fácilmente procesable por medios automáticos debido a la gran capacidad de cálculo de los ordenadores actuales.
 \item Los usuarios se acostumbran a servicios que son de coste marginal y por ende a servicios gratuitos cuyo modelo de negocio se basa en la publicidad contextual.
 \item Los gobiernos y empresas se aprovechan de toda esta información que anteriormente era imposible de recopilar, estructurar o usar de forma masiva, y ahora sí que es posible hacerlo de forma masiva y transparente.
\end{enumerate}

No es ningún secreto que empresas como Google basan su modelo de negocio en la información, y tampoco es ningún secreto (de hecho por definición una ley debe ser pública) que los Gobiernos se aprovechan de la digitalización de la información promulgando leyes como la Directiva Europea de Retención de Datos y la transposición de dicha directiva en España, la Ley de Retención de Datos, que especifica que por ejemplo debe guardarse la siguiente información sobre todas las páginas web que visitemos: URL, fecha de visita, dirección IP desde la cual se visitó. Por último y no menos importante, debido a que la información es poder, tampoco hay que descartar la posibilidad de que malhechores intenten beneficiarse de la inseguridad de nuestros datos para hacer negocios ilegales con ellos.

En un mundo cada vez más informatizado, hay una cantidad creciente y preocupante de información sobre nosotros en la red, que inevitablemente pasa por múltiples intermediarios y que no controlamos. Por supuesto, existen medios para conseguir controlar en mayor o menor medida esta información, siendo el cifrado la principal herramienta.

El cifrado consiste en el intercambio de mensajes de manera que sólo puedan ser leídos en claro por personas a las que van dirigidos y que poseen los medios para descifrarlos. Muchos protocolos de comunicación en Internet soportan cifrado, y sería del todo aconsejable que se cifrasen las comunicaciones siempre que fuese posible, no obstante no es así. Sin ir más lejos, el protocolo fundamental bajo el que funcionan las páginas web, HTTP, soporta cifrado mediante su variante HTTPS, pero un muy bajo porcentaje de páginas web lo soporta, y un porcentaje mucho menos lo utiliza por defecto.

Cuando entramos en una página web mediante HTTPS mantenemos una conversación segura con el servidor, de tal manera que la llamada comunicación cliente-servidor no resulta comprometida. Este sistema es bastante robusto y suficente como para que millones de personas realicen sus compras por Internet, se autentiquen de forma segura o administren sus cuentas mediante la banca electrónica.

No obstante cuando nos comunicamos por Internet con otras personas lo corriente no es intentar comunicarnos con personas que se encuentren en el servidor, sino con personas que se encuentran al mismo nivel que nosotros, conectadas al servidor, de tal manera que la conexión segura se establece entre nosotros y el servidor, y luego entre el servidor y la persona con la que queramos hablar, en el mejor de los casos. Los tres puntos que pueden ver el texto de la comunicación en claro son: nosotros, el servidor, y la persona con la que nos comunicamos. Sin embargo, el servidor no es el destinatario de la información y por tanto para mayor seguridad no debería de conocer el contenido de la misma.

HTTPS utiliza para establecer una comunicación segura el protocolo SSL (Secure Socket Layer). Servicios no basados en la web como el correo electrónico (con IMAP o POP) soportan el mismo método de seguridad de las comunicaciones, pero tienen el mismo problema: la comunicación no es segura de ``punto a punto'' sino de ``cliente a servidor''. Lo mismo ocurre con otros servicios de comunicación entre personas (que son los ``puntos`` que se comunican), como la mensajería instantánea o el VoIP: muchos soportan únicamente cifrado de cliente a servidor de forma nativa. 

Para resolver dicho problema, existen diferentes soluciones que implementan la seguridad punto a punto (''end-to-end security``):
\begin{itemize}
 \item Para el correo electrónico se puede utilizar PGP (Pretty Good Privacy) o su variante libre GnuPG para cifrar y firmar los mensajes. Este es un estándar utilizado e implementado por la mayoría de clientes de correo electrónico: KMail, Outlook, Thunderbird, etc.
 \item Para la mensajería instantánea se puede utilizar OTR (Off The Record), igualmente para firmar y cifrar los mensajes. OTR se ha convertido también en un estándar soportado por una plétora de clientes de mensajería instantánea, entre ellos Kopete, Pidgin, o Adium.
 \item Para el VoIP existe otro sistema criptográfico llamado ZRTP y que es una creación de Phil Zimmermann, uno de los creadores del anteriormente mencionado PGP, y que permite ser usado en la mayoría de clientes VoIP.
\end{itemize}

(Es normal que el lector no conozca algunos o todos los términos aquí empleados. Por ello en el siguiente apartado explicaremos muchos de los conceptos que subyacen en el problema que abordamos, y definiremos muchos de los términos técnicos antes mencionados.)

Uno de los principales inconvenientes que surge a la hora de utilizar un sistema seguro de comunicación es que suele ser necesario bien instalarse un software concreto para que este funcione, o bien configurarlo de forma especial. Plantearse usar cifrado o no ya es una barrera infranqueable para la mayoría de los usuarios de ordenadores puesto que no están concienciados en el problema que supone la pérdida de privacidad inherente a enviar mensajes en claro por la red, porque ni se dan cuenta de que ésto está ocurriendo ni de las repercusiones a largo o medio plazo que ésto conlleva.

Por otra parte, podría pensarse que al menos la tecnología está ahí y sólo es necesario llevar a cabo un movimiento de concienciación social en el tema de la seguridad para que empiece a usarse. En parte esto es cierto, y por ejemplo hay software como Adium (uno de los clientes de mensajería instantánea antes mencionado) que viene configurado por defecto con OTR activado de manera que utiliza lo que es conocido como ''cifrado oportunista``: si el usuario comienza una conversación con una persona que esté utilizando también un cliente de mensajería instantánea con soporte de OTR, Adium se da cuenta y automáticamente comienza una sesión de OTR segura en la que los mensajes van firmados y cifrados.

No obstante incluso si los clientes de comunicaciones de escritorio parece que cada vez son más amigables en este sentido, también debemos contar con el factor web. Y es que cada vez más aplicaciones que antes eran típicamente aplicaciones de escritorio, ahora lo son directamente web. Ésto está ocurriendo con Gmail por ejemplo, que ahora sirve tanto como cliente de correo como de mensajería instantánea. Y recientemente ahora estas aplicaciones web cada vez soportan más características, como por ejemplo videoconferencia y conversaciones de voz.

Es decir, mediante los clientes web tenemos la mayoría de las características que se encontraban en los clientes de escritorio, pero con la ventaja de que a diferencia de éstos, no hace falta instalar ni configurar ningún software (a parte del navegador) para hacerlos funcionar, y están disponibles en cualquier navegador web y sistema operativo, desde dispositivos móviles a ordenadores de sobremesa.

Si bien con los clientes de escritorio era posible mantener cierto grado de seguridad ''end-to-end``, pese a no haberse llegado a un uso generalizado de estos sistemas de cifrado de las comunicaciones, en los sistemas de comunicación via web nos encontramos con una situación desoladora, puesto que la seguridad punto a punto no se implementa de forma nativa en prácticamente ningún de estos servicios web y de hecho, como veremos a lo largo de este documento, esto es debido entre otras cosas a que no existe ninguna manera confiable y sencilla de implementarla. 

Mientras que las páginas web cada vez se utilizan más para comunicar a las personas, la privacidad online se deteriora. Tal y como mencinaban en un artículo recientemente en Slashdot \cite{onlineprivacybroken}, la industria del software no focaliza sus esfuerzos en mantener la privacidad de sus clientes porque no es lo que sus clientes no se lo piden. Y cuando  se lo piden, entonces es una ocurriencia tardía. La seguridad y privacidad son conceptos que no son fácilmente empotrables en un sistema si no ha sido diseñado para ello.

Todo esto es lo que motivó al autor a realizar su proyecto de fin de carrera sobre un tema tan controvertido. En la siguiente sección se detallarán los objetivos que se plantean en base a los motivos anteriormente expuestos.

\section{Objetivos}\label{objetivos}

Este proyecto informático tiene como principal objetivo demostrar que es posible implantar un sistema de seguridad punto a punto en las páginas web de forma que el usuario de dichas páginas web no tenga que confiar en dichas páginas web para saber que los mensajes que el usuario envíe y reciba van por un canal seguro, en el que se puede saber que sólo el destinatario de los mensajes puede leerlos y que el remitente es quien dice ser. Al utilizar el sistema de seguridad que se propone en este proyecto, el usuario únicamente debe confiar en que su navegador esté implementando los algoritmos correspondientes, como con cualquier otro software con soporte de cifrado de escritorio.

Para cumplir el mencionado objetivo, se propondrá una pequeña extensión de HTML5 y luego se implementará dicha extensión en el motor de renderizado KHTML utilizado en todo el entorno de escritorio KDE y en concreto en el navegador web Konqueror. También se expondrán las soluciones que existen actualmente como por ejemplo Hushmail o FirePGP y las ventajas y desventajas que tienen.

Es más que probable que existan otras tantas personas alrededor del globo trabajando en diferentes proyectos con el mismo objetivo que el que aquí se presenta de forma simultánea, y si no lo están haciendo ahora lo harán tarde o temprano. Implementar un esquema ''confiable'' de seguridad punto a punto en el navegador web es algo tan básico que tarde o temprano se establecerá algun tipo de estándar en los navegadores dominantes y las páginas web empezarán a utilizarlo.

Mediante este proyecto además se persigue el objetivo de concienciar a la sociedad de la relevancia de la privacidad y la seguridad en la red. No obstante el autor es consciente de las limitaciones que tiene este trabajo. Si bien el autor está convencido de haber conseguido de forma exitosa el objetivo del proyecto, no espera que todo Internet realice el salto a la tecnología propuesta de la noche a la mañana. La extensión de HTML5 que aquí se propone debido a la naturaleza de un proyecto de fin de carrera tiene también sus limitaciones y es un ''proof-of-concept`` más que otra cosa: una demostración de lo que es posible realizar, una huella más en el camino que marca el siguiente paso natural en la seguridad web tras la aparición de HTTPS, pero ni mucho menos el fin del trayecto.

Otra de las motivaciones que hay detrás de este proyecto es mostrar que un motor de renderizado actualmente bastante olvidado como KHTML aun tiene mucho que decir. La integración de KHTML con el resto del entorno de escritorio KDE hace que resulte más sencillo adentrarse en su código para alguien que conozca el funcionamiento de las librerías KDE como es el caso del autor, a la par de que debido también a dicha integración un cambio en KHTML repercutirá en todas las partes de KDE que utilizen este motor de renderizado web. En esta memoria se enseña en la sección del Manual a instalar y configurar el entorno de desarrollo de KDE para poder compilar el código, y también se explica el funcionamiento interno de KHTML puesto que es necesario conocerlo para luego poder aplicar los cambios necesarios para implementar la extensión de HTML. Se explicará dicha implementación, su funcionamiento y las medidas de seguridad que se han desarrollado.

Por último, se ha comenzó creando una aplicación de ejemplo consistente en un sencillo chat que utiliza la extensión de HTML desarrollada para poder mantener una conversación cifrada y segura de forma sencilla y transparente al desarrollador. De hecho esta aplicación de chat se ha desarrollado basándose una aplicación de chat existente creada con django y añadiéndole soporte de cifrado.

No obstante el código de la aplicación de chat en la que se basó el autor dejaba bastante que desear y al final se deshechó la idea y se optó por desarrollar una segunda aplicación de ejemplo, una extensión para la aplicación web Sweetter de microblogging. Este plugin añade soporte de envío y recepción de mensajes privados cifrados mediante GPG entre los usuarios.
% 2. Introducci�n, conceptos. arquitectura cliente-servidor, man-in-the-middle, server-in-the-middle, https
% 2. Introducción, conceptos. arquitectura cliente-servidor, man-in-the-middle, server-in-the-middle, https
\chapter{Introducción}\label{introduccion}
\section{Definiciones y abreviaturas}\label{definicionesyabreviaturas}

\begin{description}


\item[Arquitectura cliente-servidor]
Esta arquitectura consiste básicamente en un cliente que realiza peticiones a otro programa (el servidor) que le da respuesta. Aunque esta idea se puede aplicar a programas que se ejecutan sobre una sola computadora es más ventajosa en un sistema operativo multiusuario distribuido a través de una red de computadoras.

En esta arquitectura la capacidad de proceso está repartida entre los clientes y los servidores, aunque son más importantes las ventajas de tipo organizativo debidas a la centralización de la gestión de la información y la separación de responsabilidades, lo que facilita y clarifica el diseño del sistema.

La separación entre cliente y servidor es una separación de tipo lógico, donde el servidor no se ejecuta necesariamente sobre una sola máquina ni es necesariamente un sólo programa. Los tipos específicos de servidores incluyen los servidores web, los servidores de archivo, los servidores del correo, etc. Mientras que sus propósitos varían de unos servicios a otros, la arquitectura básica seguirá siendo la misma.

\item[Cifrado de clave pública]
También llamado \textbf{cifrado asimétrico}, es el método criptográfico que usa un par de claves para el envío de mensajes. Las dos claves pertenecen a la misma persona a la que se ha enviado el mensaje. Una clave es pública y se puede entregar a cualquier persona, la otra clave es privada y el propietario debe guardarla de modo que nadie tenga acceso a ella. Además, los métodos criptográficos garantizan que esa pareja de claves sólo se puede generar una vez, de modo que se puede asumir que no es posible que dos personas hayan obtenido casualmente la misma pareja de claves.

Si el remitente usa la clave pública del destinatario para cifrar el mensaje, una vez cifrado, sólo la clave privada del destinatario podrá descifrar este mensaje, ya que es el único que la conoce. Por tanto se logra la confidencialidad del envío del mensaje, nadie salvo el destinatario puede descifrarlo.

Si el propietario del par de claves usa su clave privada para cifrar el mensaje, cualquiera puede descifrarlo utilizando su clave pública. En este caso se consigue por tanto la identificación y autentificación del remitente, ya que se sabe que sólo pudo haber sido él quien empleó su clave privada (salvo que alguien se la hubiese podido robar). Esta idea es el fundamento de la firma electrónica.

\item[Criptografía]
El arte o ciencia de cifrar y descifrar información  mediante técnicas especiales y se emplea frecuentemente para permitir un intercambio de mensajes que sólo puedan ser leídos por personas a las que van dirigidos y que poseen los medios para descifrarlos.

\item[CSS, Cascading Style Sheets]
Las \textbf{hojas de estilo en cascada}, es un lenguaje usado para definir la presentación de un documento estructurado escrito en HTML o XML (y por extensión en XHTML). El W3C (World Wide Web Consortium) es el encargado de formular la especificación de las hojas de estilo que servirán de estándar para los agentes de usuario o navegadores. La idea que se encuentra detrás del desarrollo de CSS es separar la estructura de un documento de su presentación.

Por ejemplo, el elemento de HTML \verb|<h1>| indica que un bloque de texto es un encabezamiento y que es más importante que un bloque etiquetado como \verb|<h2>|. Versiones más antiguas de HTML permitían atributos extra dentro de la etiqueta abierta para darle formato (como el color o el tamaño de fuente). No obstante, cada etiqueta \verb|<h1>| debía disponer de la información si se deseaba un diseño consistente para una página y, además, una persona que lea esa página con un navegador pierde totalmente el control sobre la visualización del texto.

Cuando se utiliza CSS, la etiqueta \verb|<h1>| no debería proporcionar información sobre como va a ser visualizado, solamente marca la estructura del documento. La información de estilo separada en una hoja de estilo, especifica cómo se ha de mostrar \verb|<h1>|: color, fuente, alineación del texto, tamaño y otras características no visuales como definir el volumen de un sintetizador de voz (véase Sintetización del habla), por ejemplo.

La información de estilo puede ser adjuntada tanto como un documento separado o en el mismo documento HTML. En este último caso podrían definirse estilos generales en la cabecera del documento o en cada etiqueta particular mediante el atributo "style".

\item[End-to-end Security]
También conocida en español como \textbf{Seguridad punto a punto}, significa que los datos van firmados y/o cifrados entre aquellos que se comunican. En vez de confiar en Seguridad de la capa de transporte (transport-layer security, TLS), la seguridad punto a punto funciona a través de dominios de confianza. Esto resuelve por ejemplo el problema, cuando las sesiones TLS son finalizadas en un cortafuegos, y el DMZ es considerado diferente al dominio de confianza.

\item[GPG, Gnu Privacy Guard o GnuPG]
Es la alternativa de software libre a la suite de software criptográfico PGP. GnuPG cumple con el RFC 4880, que es la actual especificación estándar del IETF de OpenPGP. Las versiones actuales de PGP son interoperables con GnuPG y con otros sistemas OpenPGP-compatibles.

GPG es parte del proyecto de software GNU de la Free Software Fundation, y ha recibido un importante apoyo financiero del gobierno alemán. Está disponible bajo la versión 3 de la Licencia General Pública GNU.

\item[HTML, HyperText Markup Language]
En castellano lenguaje de marcas de hipertexto, es el lenguaje de marcas predominante para las páginas web. Provee un medio para crear documentos estructorados mediante la denotación de semánticas estructurales para texto como cabeceras, párrafos, listas, enlaces, citas, y otros elementos. Permite incrustar imágenes y objetos y puede usar para crear formularios interactivos. Está escrito en forma de elementos HTML que consisten en ``etiquetas'' rodeadas con ``\verb|<|'' y ``\verb|>|'' dentro del contenido de la página. Puede cargar scripts en idiomas como Javascript que afectan al comportamiento de las páginas HTML. HTML también puede incluir hojas de estilo en cascada (CSS) para definir la apariencia y distribución del texto y el resto de los elementos. El W3C es el mantenedor de ambos estándares: HTML y CSS.

\item[HTTPS, Hypertext Transfer Protocol Secure]
En español Protocolo seguro de transferencia de hipertexto, más conocido por sus siglas \textbf{HTTPS}, es un protocolo de red basado en el protocolo HTTP, destinado a la transferencia segura de datos de hipertexto, es decir, es la versión segura de HTTP.

Es utilizado principalmente por entidades bancarias, tiendas en línea, y cualquier tipo de servicio que requiera el envío de datos personales o contraseñas.

\item[HTTP, Hypertext Transfer Protocol o HTTP]
En español protocolo de transferencia de hipertexto, es el protocolo  usado en cada transacción de la World Wide Web. HTTP fue desarrollado por el World Wide Web Consortium y la Internet Engineering Task Force (IETF), colaboración que culminó en 1999 con la publicación de una serie de RFC, siendo el más importante de ellos el RFC 2616, que especifica la versión 1.1. HTTP define la sintaxis y la semántica que utilizan los elementos de software de la arquitectura web (clientes, servidores, proxies) para comunicarse. Es un protocolo orientado a transacciones y sigue el esquema petición-respuesta entre un cliente y un servidor. Al cliente que efectúa la petición (un navegador web o un spider) se lo conoce como ``user agent'' (agente del usuario). A la información transmitida se la llama recurso y se la identifica mediante un localizador uniforme de recursos (URL). Los recursos pueden ser archivos, el resultado de la ejecución de un programa, una consulta a una base de datos, la traducción automática de un documento, etc.

HTTP es un protocolo sin estado, es decir, que no guarda ninguna información sobre conexiones anteriores. El desarrollo de aplicaciones web necesita frecuentemente mantener estado. Para esto se usan las cookies, que es información que un servidor puede almacenar en el sistema cliente. Esto le permite a las aplicaciones web instituir la noción de ``sesión'', y también permite rastrear usuarios ya que las cookies pueden guardarse en el cliente por tiempo indeterminado.

\item[IMAP, Internet Message Access Protocol]
Es un protocolo de red de acceso a mensajes electrónicos almacenados en un servidor. Mediante IMAP se puede tener acceso al correo electrónico desde cualquier equipo que tenga una conexión a Internet. IMAP tiene varias ventajas sobre POP, que es el otro protocolo empleado para obtener correo desde un servidor. Por ejemplo, es posible especificar en IMAP carpetas del lado servidor. Por otro lado, es más complejo que POP ya que permite visualizar los mensajes de manera remota y no descargando los mensajes como lo hace POP.

\item[Javascript]
Es un lenguaje de scripting orientado a objetos, utilizado para acceder a objetos en aplicaciones. Principalmente, se utiliza integrado en un navegador web permitiendo el desarrollo de interfaces de usuario mejoradas y páginas web dinámicas. JavaScript es un dialecto de ECMAScript  y se caracteriza por ser un lenguaje basado en prototipos, con entrada dinámica y con funciones de primera clase. JavaScript ha tenido influencia de múltiples lenguajes y se diseñó con una sintaxis similar al lenguaje de programación Java, aunque más fácil de utilizar para personas que no programan.

Todos los navegadores modernos interpretan el código JavaScript integrado dentro de las páginas web. Para interactuar con una página web se provee al lenguaje JavaScript de una implementación del DOM. El lenguaje fue inventado por Brendan Eich en la empresa Netscape Communications, la que desarrolló los primeros navegadores web comerciales. Apareció por primera vez en el producto de Netscape llamado Netscape Navigator 2.0.

Tradicionalmente, se venía utilizando en páginas web HTML, para realizar operaciones y en el marco de la aplicación cliente, sin acceso a funciones del servidor. JavaScript se ejecuta en el agente de usuario, al mismo tiempo que las sentencias van descargándose junto con el código HTML.

En 1997 los autores propusieron JavaScript para que fuera adoptado como estándar de la European Computer Manufacturers 'Association ECMA, que a pesar de su nombre no es europeo sino internacional, con sede en Ginebra. En junio de 1997 fue adoptado como un estándar ECMA, con el nombre de ECMAScript. Poco después también como un estándar ISO.

\item[KDE]
Es una comunidad de software libre que produce un conjunto integrado de aplicaciones multiplataforma diseñadas para funcionar en Linux, FreeBSD, Windows, Solaris y Mac OS X. Es más conocido conocido por el entorno de escritorio Plasma, el entorno de escritorio que viene como el entorno de trabajo por defecto en muchas distribuciones como openSUSE, Mandriva Linux, Slackware, PCLinuxOS, Kubuntu, Arch Linux, VectorLinux, Pardus y otras. El fin del proyecto es proveer las funciones básicas de escritorio y las aplicaciones de uso diario así como las herramientas y la documentación para que los desarrolladores puedan escribir aplicaciones para el sistema. En este aspecto, el proyecto KDE sirve como un proyecto que acoge a muchas otras aplicaciones y menores proyectos que están basados en tecnología KDE, entre las que se incluyen KOffice, KDevelop, Amarok, K3B y muchas otras aplicaciones. El software KDE está basado en el toolkit Qt, aunque tambien tiene soporte para programas basados en GTK, así como para temas visuales basados en GTK. La versión original GPL de este toolkit sólo existía para la plataforma X11, pero desde el lanzamiento de Qt 4, hay versiones GPL disponibles para todas las plataformas soportadas. Esto permite al software KDE basado en Qt4 ser también ser distribuido en Microsoft Windows y Mac OS X.

\item[KHTML]
Es el motor de renderizado HTML libre desarrollado para el proyecto KDE. Fue creado para el navegador web de KDE, Konqueror. El motor fue más tarde adaptado en enero del 2003 por Apple para su navegador Safari, y la compañía devolvió todas las mejoras aplicadas sobre el código original, tal como pide la licencia. Otra compañía utilizando KHTML es YellowTAB, la cual comercializa una distribución de BeOS. KHTML fue escrito en C++ y se encuentra liberado bajo la licencia LGPL.

\item[Man in the Middle]
En criptografía, un ataque man-in-the-middle  (MitM o intermediario, en castellano) es un ataque en el que el enemigo adquiere la capacidad de leer, insertar y modificar a voluntad, los mensajes entre dos partes sin que ninguna de ellas conozca que el enlace entre ellos ha sido violado. El atacante debe ser capaz de observar e interceptar mensajes entre las dos víctimas. El ataque MitM es particularmente significativo en el protocolo original de intercambio de claves de Diffie-Hellman, cuando éste se emplea sin autenticación.


\item[OTR, Off-the-Record]
Es un protocolo criptográfico que provee cifrado fuerte para conversaciones de mensajería instantánea. OTR usa una combinación del algoritmo de clave asimétrica AES, el intercambio de claves de Diffie-Hellman, y la función de hash SHA-1. Además de autenticación y cifrado, OTR provee secreto perfecto y cifrado maleable.

La principal motivación detrás del protocolo era permitir la negación de la conversación de sus participantes al vez que se mantenía conversación confidencial, como con una conversación en la vida real. Eso es en contraste con la mayoría de herramientas de cifrado que se asemejan más a una escritura firmada en papel, que puede ser usada luego como un registro para demostrar que la comunicación ocurrió, los participantes, y el contenido de la comunicación. En la mayoría de los casos la gente usando software de cifrado no se da cuenta de esto y puede que OTR les fuese más útil. El paper inicial introductorio se llamaba ``"Off-the-Record Communication, or, Why Not To Use PGP''.

El protocolo OTR fue diseñado por los criptógrafos Ian Goldberg y Nikita Borisov. Crearon una librería para facilitar a los desarrolladores de clientes de mensajería instantánea implementar el protocolo y un proxy-OTR especial para AIM, ICQ, y .Mac, clientes que soportan proxies.


\item[OpenPGP]
La IETF se ha basado en el diseño de PGP para crear el estándar de Internet OpenPGP. Las últimas versiones de PGP son conformes o compatibles en mayor o menor medida con ese estándar. 

\item[PGP, Pretty Good Privacy]
En español privacidad bastante buena, es un programa desarrollado por Phil Zimmermann y cuya finalidad es proteger la información distribuida a través de Internet mediante el uso de criptografía de clave pública, así como facilitar la autenticación de documentos gracias a firmas digitales.

PGP originalmente fue diseñado y desarrollado por Phil Zimmermann en 1991. El nombre está inspirado en el del colmado Ralph's Pretty Good Grocery de Lake Wobegon, una ciudad ficticia inventada por el locutor de radio Garrison Keillor.

\item[PKI, Public Key Infrastructure]
En criptografía, una infraestructura de clave publica es una combinación de hardware y software, políticas y procedimientos de seguridad que permiten la ejecución con garantías de operaciones criptográficas como el cifrado, la firma digital o el no repudio de transacciones electrónicas.

El término PKI se utiliza para referirse tanto a la autoridad de certificación y al resto de componentes, como para referirse, de manera más amplia y a veces confusa, al uso de algoritmos de clave pública en comunicaciones electrónicas. Este último significado es incorrecto, ya que no se requieren métodos específicos de PKI para usar algoritmos de clave pública.


\item[POP3]
En informática se utiliza el Post Office Protocol (Protocolo de la oficina de correo) en clientes locales de correo para obtener los mensajes de correo electrónico almacenados en un servidor remoto. Es un protocolo de nivel de aplicación en el Modelo OSI.

\item[Privacidad]
La privacidad puede ser definida como el ámbito de la vida personal de un individuo que se desarrolla en un espacio reservado y debe mantenerse confidencial.

Aunque privacy deriva del latín privatus, privacidad se ha incorporado a nuestra lengua en los últimos años a través del inglés, por lo cual el término es rechazado por algunos como un anglicismo, alegando que el término correcto es intimidad, y en cambio es aceptado por otros como un préstamo lingüístico válido.

Según el Diccionario de la lengua española de la Real Academia Española - DRAE, privacidad se define como "ámbito de la vida privada que se tiene derecho a proteger de cualquier intromisión" e intimidad se define como "zona espiritual íntima y reservada de una persona o de un grupo, especialmente de una familia".

El desarrollo de la Sociedad de la Información y la expansión de la Informática y de las Telecomunicaciones plantea nuevas amenazas para la privacidad que han de ser afrontadas desde diversos puntos de vista: social, cultural, legal, tecnológico...

\item[S/MIME, Secure / Multipurpose Internet Mail Extensions]
En español Extensiones de Correo de Internet de Propósitos Múltiples / Seguro, es un estándar para criptografía de clave pública y firmado de correo electrónico encapsulado en MIME.

\item[SIP, Session Initiation Protocol]
Es un protocolo desarrollado por el grupo de trabajo MMUSIC del IETF con la intención de ser el estándar para la iniciación, modificación y finalización de sesiones interactivas de usuario donde intervienen elementos multimedia como el video, voz, mensajería instantánea, juegos en línea y realidad virtual.

La sintaxis de sus operaciones se asemeja a las de HTTP y SMTP, los protocolos utilizados en los servicios de páginas Web y de distribución de e-mails respectivamente. Esta similitud es natural ya que SIP fue diseñado para que la telefonía se vuelva un servicio más en Internet.[1]

En noviembre del año 2000, SIP fue aceptado como el protocolo de señalización de 3GPP y elemento permanente de la arquitectura IMS (IP Multimedia Subsystem). SIP es uno de los protocolos de señalización para voz sobre IP, otro es H.323 y IAX actualmente IAX2.

\item[SSL, Secure Sockets Layer]
En español Protocolo de Capa de Conexión Segura- (SSL) y Transport Layer Security -Seguridad de la Capa de Transporte- (TLS), su sucesor, son protocolos criptográficos que proporcionan comunicaciones seguras por una red, comúnmente Internet.

Existen pequeñas diferencias entre SSL 3.0 y TLS 1.0, pero el protocolo permanece sustancialmente igual. El término "SSL" según se usa aquí, se aplica a ambos protocolos a menos que el contexto indique lo contrario.

\item[Seguridad]
El término seguridad proviene de la palabra securitas del latín. Cotidianamente se puede referir a la seguridad como la ausencia de riesgo o también a la confianza en algo o alguien. Sin embargo, el término puede tomar diversos sentidos según el área o campo a la que haga referencia.

La seguridad es un estado de ánimo, una sensación, una cualidad intangible. Se puede entender como un objetivo y un fin que el hombre anhela constantemente como una necesidad primaria.

Según la pirámide de Maslow, la seguridad en el hombre ocupa el segundo nivel dentro de las necesidades de déficit.

\item[VoIP]
Voz sobre Protocolo de Internet, también llamado Voz IP, VozIP, VoIP (por sus siglas en inglés), es un grupo de recursos que hacen posible que la señal de voz viaje a través de Internet empleando un protocolo IP (Protocolo de Internet). Esto significa que se envía la señal de voz en forma digital, en paquetes, en lugar de enviarla en forma digital o analógica, a través de circuitos utilizables sólo para telefonía como una compañía telefónica convencional o PSTN (sigla de Public Switched Telephone Network, Red Telefónica Pública Conmutada).

Los Protocolos que se usan para enviar las señales de voz sobre la red IP se conocen como protocolos de Voz sobre IP o protocolos IP. Estos pueden verse como aplicaciones comerciales de la "Red experimental de Protocolo de Voz" (1973), inventada por ARPANET.

El tráfico de Voz sobre IP puede circular por cualquier red IP, incluyendo aquellas conectadas a Internet, como por ejemplo las redes de área local (LAN).

\item[W3C, World Wide Web Consortium]
Es un consorcio internacional que produce recomendaciones para la World Wide Web. Está dirigida por Tim Berners-Lee, el creador original de URL (Uniform Resource Locator, Localizador Uniforme de Recursos), HTTP (HyperText Transfer Protocol, Protocolo de Transferencia de HiperTexto) y HTML (Lenguaje de Marcado de HiperTexto) que son las principales tecnologías sobre las que se basa la Web.

\item[XML Encryption, XML-Enc]
Es una especificación, gobernada por una recomendación del W3C, que define como cifrar los contenidos de un elemento XML.

Aunque el cifrado XML puede usarse para cifrar cualquier tipo de datos, es sin embargo conocido como ``Cifrado XML'' porque un elemento XML (que es bien un elemento EncryptedData o EncryptedKey) contiene o se refiere al texto cifrado, a la información de la clave, y a los algoritmos.

Ambos XML Signature y XML Encryption usan el elemento KeyInfo, que aparece como hijo de los elementos SignedInfo, EncryptedData, o EncryptedKey y provee información a un destinatario sobre el material de claves usado para validar una firma o descrifrar datos. No obstante el elemento KeyInfo es opcional: puede ir adjunto en el mensaje o ser obtenido por un canal seguro.

\item[XML Signature, XMLDsig, XML-DSig, XML-Sig]

Es una recomendación del W3C que define una sintaxis XML para firmas digitales. Funcionalmente, tiene mucho en común con PKCS\#7 pero es mucho más extensible y más enfocado hacia la firma de documentos XML. Se usa en varias tecnología web como en SOAP, SAML y otros.

Las firmas XML pueden usarse firmar documento XML, pero cualquier cosa que sea accesible mediante una URL puede realmente ser firmada. Una firma XML usada para firmar un recurso fuera de un documento XML se llama una firma separada (detached). Si se utiliza para firmar una parte del documento que la contiene, se llama una firma envuelta (enveloped). Si contiene los datos firmados dentro de sí mismo se llama una firma envolvente (enveloping).

\item[XMPP, Extensible Messaging and Presence Protocol, Jabber]
El Protocolo extensible de mensajería y comunicación de presencia anteriormente llamado Jabber, es un protocolo abierto y extensible basado en XML, originalmente ideado para mensajería instantánea.

Con el protocolo XMPP queda establecida una plataforma para el intercambio de datos XML que puede ser usada en aplicaciones de mensajería instantánea. Las características en cuanto a adaptabilidad y sencillez del XML son heredadas de este modo por el protocolo XMPP.

A diferencia de los protocolos propietarios de intercambio de mensajes como ICQ, Y! y Windows Live Messenger, se encuentra documentado y se insta a utilizarlo en cualquier proyecto. Existen servidores y clientes libres que pueden ser usados sin coste alguno. Este es el protocolo que seleccionó Google para su servicio de mensajería Google Talk.

\item[ZRTP]
ZRTP es una extensión de Real-time Transport Protocol (RTP) que describe el establecimiento de un intercambio Diffie-Hellman de claves para el Secure Real-time Transport Protocol (SRTP). Fue enviado al IETF por Phil Zimmermann, Jon Callas y Alan Johnston el 5 de marzo de 2006. Session Initiation Protocol (SIP) es un estándar VoIP.

ZRTP se describe en el Internet-Draft  como un ``protocolo de acuerdo de claves que realiza un intercambio de claves Diffie-Hellman durante el establecimiento en banda (in-band) de una llamada en el flujo de datos Real-time Transport Protocol (RTP) que ha sido establecido empleando otro protocolo de señalización como pueda ser Session Initiation Protocol (SIP). Esto genera un secreto compartido que es usado para generar las claves y el salt para una sesión de Secure RTP (SRTP).'' Una de las funcionalidades de ZRTP es que no requiere el intercambio previo de otros secretos compartidos o una Infrastructura de Clave Pública (PKI), a la vez que evita ataques de man in the middle. Además, no delegan en la señalización SIP para la gestión de claves ni en ningún servidor. Soporta cifrado oportunista detectando automáticamente si el cliente VoIP del otro lado soporta ZRTP.

ZRTP puede usarse con cualquier protocolo de señalización como SIP, H.323, Jabber, y Peer-to-Peer SIP. ZRTP es independiente de la capa de señalización, puesto que realiza toda su negociación de claves dentro del flujo de datos RTP.

\end{description}

% 3. An�lisis de antecedentes y aportaci�n realizada
% 3.1 Exposici�n del problema
% 3.1.1 Antecedentes Aplicaciones de Escritorio (clientes de correo y mensajer�a instant�nea con PGP/OTR)
% 3.1.2 Antecedentes Aplicaciones web (actualidad, HTTPS)
% 3.1.3 Hushmail (objetivo y limitaciones)
% 3.1.4 XML Signature (objetivo y limitaciones)
% 3.2 Soluci�n propuesta
% 3.2.1 Posibles soluciones barajadas
% 3.2.2 Soluci�n elegida
% 3. Análisis de antecedentes y aportación realizada
\chapter{Análisis de antecedentes y aportación realizada}\label{antecedentes}

En primer lugar en este apartado es recalcable reseñar que en la actualidad no existe ningún antecedente de extensión HTML que permita realizar un cifrado punto a punto de manera segura. El único desarrollo existente parecido es FireGPG, que sin embargo más que una extensión de HTML era una extensión para Firefox de Javascript, lo cual implica que el usuario debe dar un mayor grado de confianza al servidor web.

Otro antecedente de este tipo de seguridad en la web ha sido Hushmail, un servicio web que implementaba el cifrado en un plugin Java aportado por el propio servidor. No obstante la seguridad de este sistema se puso en entredicho cuando la NSA \cite{thehushmailreport} por requisito legal forzó Hushmail a cambiar su plugin Java para permitirle el acceso al texto en claro de ciertos usuarios.

Por diferentes razones, anteriores intentos de utilizar cifrado punto a punto en navegadores web han fallado. Bien lo han implementado en un navegador web pero sin miras a una estandarización e implementando la solución a nivel de Javascript y no HTML, como es el caso de FireGPG, o bien el cifrado es implementado mediante Javascript o Java por el servidor, servidor en el cual no confiamos o no debemos confiar porque forma parte de la cadena de transporte del mensaje y no es su destinatario.

El presente proyecto fue presentado por el autor de la memoria al profesor con el cual realizó el proyecto de fin de carrera porque le pareció crucial que la seguridad en la web avance en la dirección correcta, permitiendo que las aplicaciones web no avancen arrastrando sin embargo la carga de la inseguridad o falta de seguridad. Este proyecto intenta arrojar luz sobre este tema y presentar una primera aproximación al problema. El principal objetivo y la principal aportación realizada es conseguir hacer sonar la alarma para que los internautas se den cuenta del problema del ``Server in the middle'', a la par que se brinda una solución o más bien se muestra un camino que puede dar con dicha solución a dicho problema.

% 4 Requisitos: qu� buscamos que tenga nuestra extensi�n e implementaci�n
\chapter{Análisis de requisitos}\label{requisitos}

En este capítulo describiremos cuales son los requisitos específicos que buscamos que vaya a cumplir nuestro proyecto. El objetivo a nivel de programación es conseguir extender HTML de la forma más sencilla posible para soportar seguridad de punto a punto, implementar dicha extensión en el motor de renderizado KHTML de KDE. También será necesario crear un página web de ejemplo que utilice las características que brinda esta nueva extensión de manera que se pueda ver fácilmente cual es el posible uso útil de dicha extensión.

La manera específica en que se consigue implementar este sistema de seguridad no será abordada en este capítulo sino en el siguiente, en el que se explica las soluciones que hay para abordar el problema, para luego explicar aquella por la cual optamos y las razones que respaldan esa decisión. También es relevante mencionar aquí el capítulo que le sigue, que es el de Implementación, donde se detalla el funcionamiento interno de KHTML para luego continuar explicando cómo se ha modificado KHTML para conseguir implementar la extensión propuesta, y qué medidas de seguridad se han adoptado.

Esta extensión de HTML debe permitir a los desarrolladores de sitios web utilizar seguridad de punto a punto sin que los usuarios tengan que confiar en que el sitio web utilice el cifrado de forma correcta, sino que por el contrario el usuario sólo deba confiar en su navegador web. El cifrado y descifrado debe realizarse por el navegador y la página web no debe nunca poder acceder al texto en claro. El internauta que navegue por páginas web que permitan la comunicación segura mediante este nuevo sistema debe poder de alguna manera estandarizada reconocer cuando un mensaje está siendo enviado de manera segura y conocer en todo momento a quien se está enviando el mensaje. Igualmente, debe poder reconocer visualmente cuándo un mensaje ha sido recibido de forma segura, y comprobar los datos del mensaje.

Debido a que esta extensión tiene como objetivo aumentar la seguridad de los Interanutas a la hora de utilizar Internet como una herramienta de comunicación, uno de los requisitos que debemos cumplir es implementar las medidas de seguridad necesarias para que un sitio web malicioso no pueda ni acceder al texto en claro saltándose las medidas de seguridad que implemente el navegador, ni engañar al usuario haciéndole creer que el mensaje que esté recibiendo o enviando lo esté haciendo de forma segura.

% 5 An�lisis temporal del desarrollo
\chapter{Análisis temporal y costes de desarrollo}\label{anatemporal}

\section{Análisis temporal}

En los apartados anteriores se ha descrito fielmente todo el proceso que ha dado como resultado la implementación de la extensión de HTML. Una vez cubierta la descripción del proceso, pasaremos a tratar en detalle la planificación seguida para la realización de este proyecto, así como un desglose de las labores más importantes. Dicha tarea la desarrollaremos a lo largo del capítulo que nos ocupa.

Para comenzar con el análisis temporal, hemos de distinguir todas las etapas que nos han ocupado durante la   elaboración de este proyecto. Un desglose de estas etapas sería:

\begin{description}
 \item[Aprendizaje y documentación sobre KHTML] El autor del proyecto tenía buen conocimiento de las tecnologías que se usan en KDE como son C++, Qt, parte de kdelibs o CMake. No obstante desconocía por completo el código de khtml y su funcionamiento interno, y tampoco conocía la librería libkleo que sirve de interfaz para todos los asuntos relacionados con GPG. Lo primero que tuvo que hacer es estudiar el código de khtml para conocer la viabilidad del proyecto y cómo abordarlo, así como buscar una librería que sirviese para manejar GPG y aprender a usarla.
 \item[Creación de kleocypher] Tras decir que iba a usar libkleo, me dediqué primero a conocer esta herramienta de ejemplo que hace uso de ella y establecer cómo iba a detectar con CMake la disponibilidad de la librería.
 \item[Implementación del elemento input cifrado] Este fue el primer gran paso a la hora de implementar la extensión de HTML. Fue bastante costoso temporalmente hablando porque fue mi primera contribución al código de khtml y no tenía soltura con él, pero el resultado fue satisfactorio.
 \item[Implementación del elemento div cifrado] Después de unos meses habiendo dejado el proyecto de lado por motivos laborales, lo retomé para implementar lo que quedaba de la extensión html, que es el elemento div cifrado. Debido al tiempo que había pasado, tuve que volver a ponerme al día además de aprender más sobre khtml por la dificultad añadida que entrañaba la implementación del elemento div.
 \item[Aplicación de prueba y testeo de la extensión] El fin del proyecto es en gran medida demostrar que la extensión propuesta es válida y puede ser útil, y la mejor manera de demostrarlo fue creando una aplicación web, en concreto un chat basado en django, que lo utilizase. Además esta aplicación sirvió como banco de pruebas de la extensión.
 \item[Elaboración de los capítulos de la memoria] Durante esta fase hemos realizado la redacción de los capítulos que conforman la memoria de este proyecto.
 \end{description}

La planificación temporal consistirá en una estimación del tiempo dedicado a cada una de las fasesde desarrollo del proyecto. Para ello asignaremos a cada una de esas fases las dos estimaciones siguientes:
    
\begin{description}
 \item[Estimación Inicial] la cual es empleada en los inicios del desarrollo. Suele ser poco exacta, pero se usan como primera aproximación para la viabilidad del proyecto.

 \item[Estimación Final] la cual expresa la duración y el esfuerzo real empleado.
\end{description}
 
Para realizar una evaluación de la exactitud de la estimación se realiza una comparación de los valores reales  con los valores estimados, teniendo en cuenta el error relativo medio. Tomando $RE =\frac{A - E}{A}$, donde A  representa el valor real y E el valor estimado previamente, calculamos el error relativo medio mediante la expresión:
    
\begin{displaymath}
\left(\frac{1}{n}\right) \sum_{i=1}^{n} RE   
\end{displaymath}

Las estimaciones de cada fase se han realizado en días, considerando una dedicación del autor de una media de 4 horas al día. Dicho esto obtendríamos las estimaciones mostradas en la Tabla~\ref{tab:planif}.                                       Analizando los datos de la misma obtenemos un error relativo medio de 14.02 \%.                                                     
                                                                
                                                                
% Aprendizaje y documentación sobre KHTML
% Creación de kleocypher
% Implementación del elemento input cifrado
% Implementación del elemento div cifrado
% Aplicación de prueba y testeo de la extensión
% Elaboración de los capítulos de la memoria
\begin{table}
    \centering
        \begin{tabular}{|l|r|r|r|}
            \hline
            \textbf{Tarea} & \textbf{Est. Inicial} & \textbf{Est. Final} & \textbf{RE}\\ \hline \hline
             Aprendizaje y documentación sobre KHTML  & 25 días & 36 días & 30.5 \%  \\
             Creación de kleocypher & 18 días & 19 días & 5.3 \% \\
             Implementación del elemento input cifrado & 22 días & 32 días & 30.1 \%  \\
             Implementación del elemento div cifrado & 31 días & 42 días & 30.9 \%  \\
             Aplicación de prueba y testeo de la extensión & 12 días & 14 días & 14.3  \% \\
             Elaboración capítulos memoria & 8 días & 10 días & 20 \% \\
             Elaboración de los capítulos de la memoria  & 5 días & 5 días & 0 \% \\       
            \hline
        \end{tabular}
    \caption{Planificación del proyecto}
    \label{tab:planif}
\end{table}



\section{Costes de desarrollo}

En esta sección estudiaremos detalladamente los costes que han supuesto la creación del proyecto. Comenzaremos diciendo que no ha sido necesaria inversión alguna destinada a la adquisición de dispositivos específicos, hardware o software. Este hecho ha supuesto una gran ventaja a la hora de elaborar el proyecto que nos  ocupa, ya que los costes se centrarán unicamente en los campos de costes de personal y costes indirectos.
   
\subsection{Inversiones}

En lo que respecta a la inversión para la adquisición de dispositivos específicos, tenemos que decir que en nuestro caso dicha inversión ha sido nula. Esto se debe a que no requeriamos de ningún material específico para materializar en khtml.


\subsection{Costes de software}

En lo que respecta los costes de software usado para el desarrollo del proyecto, he de indicar primeramente que todo el software empleado es free software (software libre) disponible en Internet de forma gratuita. Esto supone un coste cero en lo que respecta a este apartado.
    
Además del coste cero, el sofware libre también nos proporciona muchas otras ventajas. Este tipo de software  brinda libertad a los usuarios sobre su producto adquirido y por tanto, una vez obtenido, puede ser usado, copiado, estudiado, modificado y redistribuido libremente. Según la Free Software Foundation, el software libre se refiere a la libertad de los usuarios para ejecutar, copiar, distribuir, estudiar, cambiar y mejorar el software; de modo más preciso, se refiere a cuatro libertades de los usuarios del software: la libertad de usar el programa, con cualquier propósito; de estudiar el funcionamiento del programa, y adaptarlo a las necesidades; de distribuir copias, con lo que puede ayudar a otros; de mejorar el programa y hacer públicas las mejoras, de modo que toda la comunidad se beneficie. El software libre suele estar disponible gratuitamente, o al precio de coste de la distribución a través de otros medios; sin embargo no es obligatorio que sea así, por ende no hay que asociar la idea de software libre a software gratuito.  
     
Entre el software usado podemos citar entre otros:
    
\begin{itemize}
    \item Arch Linux, como distribución Linux.
    \item KDE, como sistema base sobre el cual se ha desarrollado.
    \item Kdevelop, entorno de desarrollo utilizado.
    \item CMake, como sistema de compilación.
    \item Kile, como editor \LaTeX{} sobre Linux.
    \item GDB, como depurador de software.
    \item Okular como visor de documentos PDF.
\end{itemize}

\subsection{Costes de hardware}

En el proceso de desarollo no ha sido necesaria la adquisición de equipos informáticos ni de ninguna clase de hardware que no tuviese ya disponible. El hardware utilizado es el portátil personal del autor. 

\subsection{Costes indirectos}

Dentro de esta sección dedicada a los costes indirectos incluiremos el gasto producido por material consumible (CD, impresora, papel, etc), luz consumida, limpieza de la zona de desarrollo, etc. He considerado adecuado que un incremento del 5 \% en el presupuesto total sería representativo de este gasto.
    
\subsection{Costes de personal}

Para calcular el coste personal he utilizado como referencia los sueldos brutos estipulados mínimos fijados en el BOE del 21 de Marzo de 2010, en el Anexo III. El sueldo anual para un Titulado de grado medio es de 14637.56 euros ó 6.53 $\frac{euros}{hora}$. 

También es necesario tener en cuenta el número total de horas dedicadas a la elaboración de este proyecto. Dicho dato tiene un valor de 672 horas, representado por el trabajo realizado por una persona durante 168 días dedicando una media de 4 horas/día.
    
De esta forma el coste personal supondrá una cuantía de:
    
\begin{center}
$672 horas \cdot 6.53 \frac{euros}{hora} = 4388.16 euros$.
\end{center}

\subsection{Presupuesto}
    
Haciendo uso de los costes especificados anteriormente, obtendremos el presupuesto mostrado en la Tabla~\ref{tab:presup}.

\begin{table}
    \centering
        \begin{tabular}{|p{7cm}|r|}
        \hline
        Inversiones & 0 euros \\
        Costes Software & 0 euros \\
        Costes Hardware & 0 euros \\
        Costes Personal & 4388.16 euros \\
        Costes Indirectos & 319.31 euros \\ \hline  
        \textbf{TOTAL} & 4707.47 euros \\   
        \hline  
        \end{tabular}
    \caption{Presupuesto del proyecto}
    \label{tab:presup}
\end{table}

% 6. Pruebas y conclusiones
\chapter{Pruebas y conclusiones}\label{pruebas}

\section{Pruebas}

	
\section{Conclusiones}\label{conclusiones} 


% 7. Manuales para empaquetadores, desarrolladores y usuarios
\chapter{Manual}\label{manual}

\section{Manual de usuario}

Esta es la documentación del proyecto que implementa una extensión HTML de cifrado punto a punto para el motor de renderizado KHTML de KDE. Este capítulo está destinado para que aprendas cómo utilizar la extensión y poder crear tu propia aplicación web con ella.

Para poder hacer uso de esta extensión es necesario parchear la librería khtml. Explicamos en la siguiente sección cómo realizar este importante paso.

\section{Introducción}

La extensión está desarrollada sobre la versión de desarrollo de la librería khtml puesto que si se hiciese sobre la versión estable sería más difícil continuar su desarrollo al entrar en conflicto el parche con las novedades incluídas en próximas versiones. Concretamente la revisión de KDE con la que es funcional el parche es la 1132763. Como requisito indispensable para poder hacer uso de la extensión actualmente se encuentra compilar la versión de desarrollo de KDE y luego parchear kdelibs.

Se explicará cómo compilar y configurar el entorno de desarrollo de KDE para la distribución Arch Linux. Es posible por supuesto configurar KDE para que pueda ser usado en otras distribuciones y los pasos necesarios para hacerlo se detallan en el wiki para desarrolladores de KDE, techbase \cite{build_kde4}.

He de aclarar que esta no es la única forma de compilar KDE. También es posible hacer uso de por ejemplo el script kdesvn-build que facilita mantenerse actualizado a la última versión, y también es posible utilizar una cuenta a parte para el desarrollo de KDE. Todas esas posibilidades se explican en techbase \cite{build_kde4}.

\section{Instalar paquetes requeridos}

Para instalar las dependencias necesarias para compilar KDE en Arch Linux, hay que ejecutar el siguiente comando con permisos de superusuario:

\begin{verbatim}
 pacman -Sy subversion bzip2 libxslt libxml2 libjpeg \
           libungif shared-mime-info mesa boost dbus \
           openssl pkgconfig xine-lib clucene redland \
           gpgme hal cmake qt qca libical lcms \
           automoc4 akonadi eigen taglib soprano \
           strigi qimageblitz phonon kdesdk git
\end{verbatim} 

Es necesario tener activado el repositorio [extra] para poder instalar dichas dependencias, entre las que incluyen las librerías incluidas en kdesupport. También he incluido git como dependencia porque lo necesitaremos para descargar el código del presente proyecto.

\section{Configuración del entorno}

Configuraremos el sistema para que convivan tanto la posible instalación de KDE estable como la versión ``trunk'' o de desarrollo. Asumiremos que todo lo relativo a la versión de desarrollo de KDE lo albergaremos en el directorio ``proyectos/kde4''. Crearemos un script en bash llamado ``environment.sh'' en ese directorio, con permisos de ejecución, y que contenga las siguientes líneas:

\begin{verbatim}
export KDEDIR=$HOME/proyectos/kde4
export KDETMP=/tmp/$USER-kde4

mkdir -p $KDETMP
export KDEVARTMP=/var/tmp/$USER-kde4
export KDEDIRS=$KDEDIR
export PKG_CONFIG_PATH=$KDEDIR/lib/pkgconfig:$PKG_CONFIG_PATH
export KDEDIRS=$KDEDIR

export PATH=$QTDIR/bin:$KDEDIR/bin:$PATH
export YACC='byacc -d'
export LD_LIBRARY_PATH=$QTDIR/lib/:$KDEDIR/lib/:$LD_LIBRARY_PATH
export CMAKE_LIBRARY_PATH=$KDEDIR/lib/:$CMAKE_LIBRARY_PATH
export CMAKE_INCLUDE_PATH=$KDEDIR/include:$CMAKE_INCLUDE_PATH
export CMAKE_PREFIX_PATH=$KDEDIR:$CMAKE_PREFIX_PATH

export KDE_BUILD=$KDEDIR/src/build/
export KDE_SRC=$KDEDIR/src/
export KDEHOME=$HOME/.kde-trunk

export KDE_COLOR_DEBUG=1
export QTEST_COLORED=1

 
##
# A function to easily build the current directory of KDE.
#
# This builds only the sources in the current ~/{src,build}/KDE subdirectory.
# Usage: cs KDE/kdebase && cmakekde
#   will build/rebuild the sources in ~/src/KDE/kdebase
#
function cmakekde {
        if test -n "$1"; then
                # srcFolder is defined via command line argument
                srcFolder=$1
        else
                # get srcFolder for current dir
                srcFolder=`pwd | sed -e s,$KDE_BUILD,$KDE_SRC,`
        fi
        # we are in the src folder, change to build directory
        # Alternatively, we could just use makeobj in the commands below...
        if [ "$srcFolder" = `pwd` ]; then
                cb
        fi
        # to enable tests, add -DKDE4_BUILD_TESTS=TRUE to the next line.
        # you can also change "debugfull" to "debug" to save disk space.
        # added "nice make..." to allow the user to work on the box while
        # compiling
        nice -n 15 cmake $srcFolder -DKDE4_BUILD_TESTS=TRUE -DCMAKE_INSTALL_PREFIX=$KDEDIR \
               -DPYTHON_SITE_PACKAGES_DIR:PATH=~/.local/lib/python2.6/site-packages \
        -DCMAKE_BUILD_TYPE=debugfull && \
        make -j2 VERBOSE=1 && \
        make install;
}
 
##
# A function to easily change to the build directory.
# Usage: cb KDE/kdebase
#   will change to $KDE_BUILD/KDE/kdebase
# Usage: cb
#   will simply go to the build folder if you are currently in a src folder
#   Example:
#     $ pwd
#     /home/user/src/KDE/kdebase
#     $ cb && pwd
#     /home/user/build/KDE/kdebase
#
function cb {
        # Make sure build directory exists.
        mkdir -p $KDE_BUILD
 
        # command line argument
        if test -n "$1"; then
                cd $KDE_BUILD/$1
                return
        fi
        # substitute src dir with build dir
        dest=`pwd | sed -e s,$KDE_SRC,$KDE_BUILD,`
        if test ! -d $dest; then
                # build directory does not exist, create
                mkdir -p $dest
        fi
        cd $dest
}
 
##
# Change to the source directory.  Same as cb, except this
# switches to $KDE_SRC instead of $KDE_BUILD.
# Usage: cs KDE/kdebase
#       will change to $KDE_SRC/KDE/kdebase
# Usage: cs
#   will simply go to the source folder if you are currently in a build folder
#   Example:
#     $ pwd
#     /home/user/build/KDE/kdebase
#     $ cs && pwd
#     /home/user/src/KDE/kdebase
#
function cs {
        # Make sure source directory exists.
        mkdir -p $KDE_SRC
 
        # command line argument
        if test -n "$1"; then
                cd $KDE_SRC/$1
        else
                # substitute build dir with src dir
                dest=`pwd | sed -e s,$KDE_BUILD,$KDE_SRC,`
                if [ $dest = `pwd` ]; then
                cd $KDE_SRC
                else
                cd $dest
                fi
        fi
}
 
export DISPLAY=:0.0
\end{verbatim}

\section{Compilación de KDE}

Ahora pasaremos a compilar todo el sistema base KDE que usaremos. Pese a que por si acaso las instalamos anteriormente, para asegurarnos que las librerías incluidas en el paquete kdesupport están suficientemente actualizadas también las compilaremos. Los comandos que debemos ejecutar en la consola con nuestro usuario son los siguientes:

\begin{verbatim}
cd ~/proyectos/kde4
. environment.sh # carga las variables de entorno y funciones útiles usadas para compilar KDE_BUILD
cs # entra en el directorio ~/proyectos/kde4/src y lo crea si no existiese.

# Ahora descargaremos el software que usaremos de KDE:
svn checkout svn://anonsvn.kde.org/home/kde/trunk/KDE/kdesupport 
svn checkout svn://anonsvn.kde.org/home/kde/trunk/KDE/kdelibs
svn checkout svn://anonsvn.kde.org/home/kde/trunk/KDE/kdebase
svn checkout svn://anonsvn.kde.org/home/kde/trunk/KDE/kdepimlibs
svn checkout svn://anonsvn.kde.org/home/kde/trunk/KDE/kdepim

# y ahora lo compilaremos e instalaremos:
cd kdesupport; cmakekde; cd ..
cd kdelibs; cmakekde; cd ..
cd kdebase; cmakekde; cd ..
cd kdepimlibs; cmakekde; cd ..
cd kdepim; cmakekde; cd ..
\end{verbatim}

Ya tenemos todo lo necesario de KDE compilado. Hemos compilado e instalado khtml y una navegador que lo utiliza, konqueror. Podemos comprobar que eso es así ejecutando el comando which konqueror, que debería de devolver algo así:

\begin{verbatim}
$ which konqueror
/home/edulix/proyectos/kde4/bin/konqueror 
\end{verbatim}

\section{Compilación del proyecto}

Ahora proseguiremos aplicando el parche a khtml y recompilando khtml con el parche:

\begin{verbatim}
git clone http://github.com/edulix/pfc.git # descargamos el código del presente proyecto
cd kdepim/libkleo
patch -p0 < ../../pfc/find_kleo.patch # parcheamos libkleo
make install # recompilamos y reinstalamos libkleo

cd ../../kdelibs/khtml
patch -p0 < ../../pfc/khtml.patch # parcheamos khtml
make install # recompilamos y reinstalamos khtml
\end{verbatim}

Ha resultado necesario recompilar antes libkleo para que pueda accederse a dicha librería externamente, en nuestro caso desde khtml. Luego hemos recompilado y reinstalado khtml con el parche que implementa el presente proyecto aplicado. Y con esto ya hemos terminado: tenemos un sistema funcional en el que podemos utilizar la nueva extensión HTML descrita en esta memoria en aplicaciones web seguras.

El navegador que utilizaremos para usar la extensión de khtml es Konqueror. De aquí en adelante siempre que vayamos a ejecutar konqueror, previamente arrancaremos un terminal y configuraremos el entorno si no lo hemos hecho antes en ese terminal. Ejemplo:

\begin{verbatim}
$ . ~/proyectos/kde4/environment.sh
$ which konqueror # comprobamos que el binario konqueror es el correcto
/home/edulix/proyectos/kde4/bin/konqueror 
$ konqueror
\end{verbatim}

El mismo comando (\verb|. ~/proyectos/kde4/environment.sh|) nos servirá para poder ejecutar cualquier aplicación de KDE que hayamos compilado, como por ejemplo kleopatra.

\section{Creación de claves GPG}

Evidentemente necesaremos tener un par de claves pública y privada para poder utilizar la extensión. Podemos usar la aplicación kleopatra para este fin. Adjuntamos una captura de pantalla del diálogo con los pasos a seguir para crear una nueva clave GPG con dicha aplicación.

\figura{1}{img/kleopatra-new-key-dialog}{Diálogo para crear una nueva clave}{kleodialog}{}

El primer paso es abrir la aplicación Kleopatra y luego pulsar Ctrl+N para abrir el diálogo de crear un nuevo certificado \ref{kleodialog}. Elegit la opción OpenPGP, escribir nuestro nombre e email, pulsar siguiente, y crear nuestra clave.

\section{Aplicación de prueba: Sweetter}

La aplicación de prueba que finalmente desarrollamos fue un plugin para el software de microblogging Sweetter. Para instalar sweetter necesitamos instalar primero sus dependencias:

\begin{verbatim}
sudo pacman -S django python-pysqlite
\end{verbatim} 

Para descargarnos la última versión de django del repositorio de gitorious donde se alberga, ejecutamos el siguiente comando en el directorio donde vayamos a descargarlo:

\begin{verbatim}
git clone git://gitorious.org/sweetter/sweetter.git
\end{verbatim}

Esto nos creará un directorio ``sweetter/'' donde se descargó la aplicación. Ahora vamos a instalarlo. Al crear la base de datos nos pedirá cierta información como si queremos crear un usuario, le diremos que sí, y daremos los datos del usuario que queramos crear. Esto lo haremos ejecutando el comando \verb|python manage.py syncdb|:

\begin{verbatim}
$ python manage.py syncdb
Creating table auth_permission
Creating table auth_group
Creating table auth_user
Creating table auth_message
Creating table django_admin_log
Creating table django_content_type
Creating table django_session
Creating table django_site
Creating table django_flatpage
Creating table ublogging_profile
Creating table ublogging_option
Creating table ublogging_post
Creating table groups_group
Creating table recoverpw_recover
Creating table karma_karma
Creating table karma_karmasweet
Creating table karma_vote
Creating table karma_log
Creating table followers_follower
Creating table privatetimeline_privatesweet
Creating table jabberbot_jabber

You just installed Django's auth system, which means you don't have any superusers defined.
Would you like to create one now? (yes/no): yes
Username (Leave blank to use 'edulix'): edulix
E-mail address: edulix@gmail.com
Password: 
Password (again): 
Superuser created successfully.
Installing index for auth.Permission model
Installing index for auth.Message model
Installing index for admin.LogEntry model
Installing index for flatpages.FlatPage model
Installing index for ublogging.Option model
Installing index for ublogging.Post model
Installing index for karma.KarmaSweet model
Installing index for karma.Vote model
Installing index for karma.Log model
Installing index for followers.Follower model
Installing index for privatetimeline.PrivateSweet model
Installing index for jabberbot.Jabber model
\end{verbatim}

Ya tenemos Sweetter instalado. Ahora podemos ejecutarlo, de la siguiente manera:

\begin{verbatim}
$ python manage.py runserver
Validating models...
0 errors found

Django version 1.1.1, using settings 'sweetter.settings'
Development server is running at http://127.0.0.1:8000/
Quit the server with CONTROL-C.
\end{verbatim}

Como podemos observar, ya tenemos el servidor sweetter ejecutándose. Si entramos en el navegador en http://127.0.0.1:8000/ veremos la página de inicio de sweetter, y nos permite autenticarnos. Para ello usaremos el usuario y contraseña que elegimos anteriormente. Haciendo clic en la barra de urls superior en ``Profile'' podemos configurar cual es el identificador de nuestra clave GPG, que utilizaremos para recibir mensajes privados cifrados.

Lamentablemente kleopatra no nos muestra para las claves GPG su identificador corto, pero verlo ejecutando el siguiente comando:

\begin{verbatim}
$ gpg --list-keys
pub   1024D/7198F146 2010-06-15 [expires: 2010-07-14]
uid                  Eduardo Robles Elvira <edulix@gmail.com>
sub   1024g/0F8F7B8D 2010-06-15 
\end{verbatim}

\figura{1}{img/sweetter-configure-gpg-key}{Configurando la clave privada del usuario desde su perfil}{sweetter-configure-gpg-key}{}

En nuestro caso el identificar sería 7198F146. Lo introducimos en la página de ``Profile'' de sweetter en el campo correspondiente, como vemos en la figura \ref{sweetter-configure-gpg-key}. 

Ahora podemos ya recibir mensajes cifrados. No hace falta crear otro usuario para que nos envíe un mensaje privado, porque podemos enviarnos un mensaje privado a nosotros mismos. Podemos por ejemplo hacer clic en la barra lateral donde pone ``Send private message'', y el recuadro donde se suele escribir el sweet se convierte en un recuadro donde podemos escribir el mensaje que nos queremos enviar. Luego, podemos ver el mensaje cifrado haciendo clic en la barra de urls superior en la opción ``Private timeline''. El resultado puede verse en \ref{private-timeline-example}.

\figura{1}{img/private-timeline-example}{El mensaje cifrado que previamente nos habíamos enviado aparece en el Private timeline}{private-timeline-example}{}

Para poder enviar un mensaje cifrado a cualquier otro usuario, debemos de tener su clave pública instalada en el sistema. Desde Kleopatra podemos importar claves públicas desde el menu ``File'', tanto desde un fichero como desde un servidor, únicamente es necesario conocer algún dato como el nombre y apellidos o la dirección de correo electrónico de aquella persona cuya clave querramos obtener.


\backmatter

\chapter{Apéndices}\label{apendices}

No ha sido necesario la inclusión de ningún apéndice.
\input{Capitulos/licencia}
\bibliographystyle{pfc}
\bibliography{pfcbib}

\end{document}
