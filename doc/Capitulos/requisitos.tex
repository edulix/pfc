\chapter{Análisis de requisitos}\label{requisitos}

En este capítulo describiremos cuales son los requisitos específicos que buscamos que vaya a cumplir nuestro proyecto. El objetivo a nivel de programación es conseguir extender HTML de la forma más sencilla posible para soportar seguridad de punto a punto, implementar dicha extensión en el motor de renderizado KHTML de KDE. También será necesario crear un página web de ejemplo que utilice las características que brinda esta nueva extensión de manera que se pueda ver fácilmente cual es el posible uso útil de dicha extensión.

La manera específica en que se consigue implementar este sistema de seguridad no será abordada en este capítulo sino en el siguiente, en el que se explica las soluciones que hay para abordar el problema, para luego explicar aquella por la cual optamos y las razones que respaldan esa decisión. También es relevante mencionar aquí el capítulo que le sigue, que es el de Implementación, donde se detalla el funcionamiento interno de KHTML para luego continuar explicando cómo se ha modificado KHTML para conseguir implementar la extensión propuesta, y qué medidas de seguridad se han adoptado.

Esta extensión de HTML debe permitir a los desarrolladores de sitios web utilizar seguridad de punto a punto sin que los usuarios tengan que confiar en que el sitio web utilice el cifrado de forma correcta, sino que por el contrario el usuario sólo deba confiar en su navegador web. El cifrado y descifrado debe realizarse por el navegador y la página web no debe nunca poder acceder al texto en claro. El internauta que navegue por páginas web que permitan la comunicación segura mediante este nuevo sistema debe poder de alguna manera estandarizada reconocer cuando un mensaje está siendo enviado de manera segura y conocer en todo momento a quien se está enviando el mensaje. Igualmente, debe poder reconocer visualmente cuándo un mensaje ha sido recibido de forma segura, y comprobar los datos del mensaje.

Debido a que esta extensión tiene como objetivo aumentar la seguridad de los Interanutas a la hora de utilizar Internet como una herramienta de comunicación, uno de los requisitos que debemos cumplir es implementar las medidas de seguridad necesarias para que un sitio web malicioso no pueda ni acceder al texto en claro saltándose las medidas de seguridad que implemente el navegador, ni engañar al usuario haciéndole creer que el mensaje que esté recibiendo o enviando lo esté haciendo de forma segura.
