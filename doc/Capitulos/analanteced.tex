% 3. Análisis de antecedentes y aportación realizada
\chapter{Análisis de antecedentes y aportación realizada}\label{antecedentes}

En primer lugar en este apartado es recalcable reseñar que en la actualidad no existe ningún antecedente de extensión HTML que permita realizar un cifrado punto a punto de manera segura. El único desarrollo existente parecido es FireGPG, que sin embargo más que una extensión de HTML era una extensión para Firefox de Javascript, lo cual implica que el usuario debe dar un mayor grado de confianza al servidor web.

Otro antecedente de este tipo de seguridad en la web ha sido Hushmail, un servicio web que implementaba el cifrado en un plugin Java aportado por el propio servidor. No obstante la seguridad de este sistema se puso en entredicho cuando la NSA \cite{thehushmailreport} por requisito legal forzó Hushmail a cambiar su plugin Java para permitirle el acceso al texto en claro de ciertos usuarios.

Por diferentes razones, anteriores intentos de utilizar cifrado punto a punto en navegadores web han fallado. Bien lo han implementado en un navegador web pero sin miras a una estandarización e implementando la solución a nivel de Javascript y no HTML, como es el caso de FireGPG, o bien el cifrado es implementado mediante Javascript o Java por el servidor, servidor en el cual no confiamos o no debemos confiar porque forma parte de la cadena de transporte del mensaje y no es su destinatario.

El presente proyecto fue presentado por el autor de la memoria al profesor con el cual realizó el proyecto de fin de carrera porque le pareció crucial que la seguridad en la web avance en la dirección correcta, permitiendo que las aplicaciones web no avancen arrastrando sin embargo la carga de la inseguridad o falta de seguridad. Este proyecto intenta arrojar luz sobre este tema y presentar una primera aproximación al problema. El principal objetivo y la principal aportación realizada es conseguir hacer sonar la alarma para que los internautas se den cuenta del problema del ``Server in the middle'', a la par que se brinda una solución o más bien se muestra un camino que puede dar con dicha solución a dicho problema.
