\chapter{Manual}\label{manual}

\section{Manual de usuario}

Esta es la documentación del proyecto que implementa una extensión HTML de cifrado punto a punto para el motor de renderizado KHTML de KDE. Este capítulo está destinado para que aprendas cómo utilizar la extensión y poder crear tu propia aplicación web con ella.

Para poder hacer uso de esta extensión es necesario parchear la librería khtml. Explicamos en la siguiente sección cómo realizar este importante paso.

\section{Introducción}

La extensión está desarrollada sobre la versión de desarrollo de la librería khtml puesto que si se hiciese sobre la versión estable sería más difícil continuar su desarrollo al entrar en conflicto el parche con las novedades incluídas en próximas versiones. Concretamente la revisión de KDE con la que es funcional el parche es la 1132763. Como requisito indispensable para poder hacer uso de la extensión actualmente se encuentra compilar la versión de desarrollo de KDE y luego parchear kdelibs.

Se explicará cómo compilar y configurar el entorno de desarrollo de KDE para la distribución Arch Linux. Es posible por supuesto configurar KDE para que pueda ser usado en otras distribuciones y los pasos necesarios para hacerlo se detallan en el wiki para desarrolladores de KDE, techbase \cite{build_kde4}.

He de aclarar que esta no es la única forma de compilar KDE. También es posible hacer uso de por ejemplo el script kdesvn-build que facilita mantenerse actualizado a la última versión, y también es posible utilizar una cuenta a parte para el desarrollo de KDE. Todas esas posibilidades se explican en techbase \cite{build_kde4}.

\section{Instalar paquetes requeridos}

Para instalar las dependencias necesarias para compilar KDE en Arch Linux, hay que ejecutar el siguiente comando con permisos de superusuario:

\begin{verbatim}
 pacman -Sy subversion bzip2 libxslt libxml2 libjpeg \
           libungif shared-mime-info mesa boost dbus \
           openssl pkgconfig xine-lib clucene redland \
           gpgme hal cmake qt qca libical lcms \
           automoc4 akonadi eigen taglib soprano \
           strigi qimageblitz phonon kdesdk git
\end{verbatim} 

Es necesario tener activado el repositorio [extra] para poder instalar dichas dependencias, entre las que incluyen las librerías incluidas en kdesupport. También he incluido git como dependencia porque lo necesitaremos para descargar el código del presente proyecto.

\section{Configuración del entorno}

Configuraremos el sistema para que convivan tanto la posible instalación de KDE estable como la versión ``trunk'' o de desarrollo. Asumiremos que todo lo relativo a la versión de desarrollo de KDE lo albergaremos en el directorio ``proyectos/kde4''. Crearemos un script en bash llamado ``environment.sh'' en ese directorio, con permisos de ejecución, y que contenga las siguientes líneas:

\begin{verbatim}
export KDEDIR=$HOME/proyectos/kde4
export KDETMP=/tmp/$USER-kde4

mkdir -p $KDETMP
export KDEVARTMP=/var/tmp/$USER-kde4
export KDEDIRS=$KDEDIR
export PKG_CONFIG_PATH=$KDEDIR/lib/pkgconfig:$PKG_CONFIG_PATH
export KDEDIRS=$KDEDIR

export PATH=$QTDIR/bin:$KDEDIR/bin:$PATH
export YACC='byacc -d'
export LD_LIBRARY_PATH=$QTDIR/lib/:$KDEDIR/lib/:$LD_LIBRARY_PATH
export CMAKE_LIBRARY_PATH=$KDEDIR/lib/:$CMAKE_LIBRARY_PATH
export CMAKE_INCLUDE_PATH=$KDEDIR/include:$CMAKE_INCLUDE_PATH
export CMAKE_PREFIX_PATH=$KDEDIR:$CMAKE_PREFIX_PATH

export KDE_BUILD=$KDEDIR/src/build/
export KDE_SRC=$KDEDIR/src/
export KDEHOME=$HOME/.kde-trunk

export KDE_COLOR_DEBUG=1
export QTEST_COLORED=1

 
##
# A function to easily build the current directory of KDE.
#
# This builds only the sources in the current ~/{src,build}/KDE subdirectory.
# Usage: cs KDE/kdebase && cmakekde
#   will build/rebuild the sources in ~/src/KDE/kdebase
#
function cmakekde {
        if test -n "$1"; then
                # srcFolder is defined via command line argument
                srcFolder=$1
        else
                # get srcFolder for current dir
                srcFolder=`pwd | sed -e s,$KDE_BUILD,$KDE_SRC,`
        fi
        # we are in the src folder, change to build directory
        # Alternatively, we could just use makeobj in the commands below...
        if [ "$srcFolder" = `pwd` ]; then
                cb
        fi
        # to enable tests, add -DKDE4_BUILD_TESTS=TRUE to the next line.
        # you can also change "debugfull" to "debug" to save disk space.
        # added "nice make..." to allow the user to work on the box while
        # compiling
        nice -n 15 cmake $srcFolder -DKDE4_BUILD_TESTS=TRUE -DCMAKE_INSTALL_PREFIX=$KDEDIR \
               -DPYTHON_SITE_PACKAGES_DIR:PATH=~/.local/lib/python2.6/site-packages \
        -DCMAKE_BUILD_TYPE=debugfull && \
        make -j2 VERBOSE=1 && \
        make install;
}
 
##
# A function to easily change to the build directory.
# Usage: cb KDE/kdebase
#   will change to $KDE_BUILD/KDE/kdebase
# Usage: cb
#   will simply go to the build folder if you are currently in a src folder
#   Example:
#     $ pwd
#     /home/user/src/KDE/kdebase
#     $ cb && pwd
#     /home/user/build/KDE/kdebase
#
function cb {
        # Make sure build directory exists.
        mkdir -p $KDE_BUILD
 
        # command line argument
        if test -n "$1"; then
                cd $KDE_BUILD/$1
                return
        fi
        # substitute src dir with build dir
        dest=`pwd | sed -e s,$KDE_SRC,$KDE_BUILD,`
        if test ! -d $dest; then
                # build directory does not exist, create
                mkdir -p $dest
        fi
        cd $dest
}
 
##
# Change to the source directory.  Same as cb, except this
# switches to $KDE_SRC instead of $KDE_BUILD.
# Usage: cs KDE/kdebase
#       will change to $KDE_SRC/KDE/kdebase
# Usage: cs
#   will simply go to the source folder if you are currently in a build folder
#   Example:
#     $ pwd
#     /home/user/build/KDE/kdebase
#     $ cs && pwd
#     /home/user/src/KDE/kdebase
#
function cs {
        # Make sure source directory exists.
        mkdir -p $KDE_SRC
 
        # command line argument
        if test -n "$1"; then
                cd $KDE_SRC/$1
        else
                # substitute build dir with src dir
                dest=`pwd | sed -e s,$KDE_BUILD,$KDE_SRC,`
                if [ $dest = `pwd` ]; then
                cd $KDE_SRC
                else
                cd $dest
                fi
        fi
}
 
export DISPLAY=:0.0
\end{verbatim}

\section{Compilación de KDE}

Ahora pasaremos a compilar todo el sistema base KDE que usaremos. Pese a que por si acaso las instalamos anteriormente, para asegurarnos que las librerías incluidas en el paquete kdesupport están suficientemente actualizadas también las compilaremos. Los comandos que debemos ejecutar en la consola con nuestro usuario son los siguientes:

\begin{verbatim}
cd ~/proyectos/kde4
. environment.sh # carga las variables de entorno y funciones útiles usadas para compilar KDE_BUILD
cs # entra en el directorio ~/proyectos/kde4/src y lo crea si no existiese.

# Ahora descargaremos el software que usaremos de KDE:
svn checkout svn://anonsvn.kde.org/home/kde/trunk/KDE/kdesupport 
svn checkout svn://anonsvn.kde.org/home/kde/trunk/KDE/kdelibs
svn checkout svn://anonsvn.kde.org/home/kde/trunk/KDE/kdebase
svn checkout svn://anonsvn.kde.org/home/kde/trunk/KDE/kdepimlibs
svn checkout svn://anonsvn.kde.org/home/kde/trunk/KDE/kdepim

# y ahora lo compilaremos e instalaremos:
cd kdesupport; cmakekde; cd ..
cd kdelibs; cmakekde; cd ..
cd kdebase; cmakekde; cd ..
cd kdepimlibs; cmakekde; cd ..
cd kdepim; cmakekde; cd ..
\end{verbatim}

Ya tenemos todo lo necesario de KDE compilado. Hemos compilado e instalado khtml y una navegador que lo utiliza, konqueror. Podemos comprobar que eso es así ejecutando el comando which konqueror, que debería de devolver algo así:

\begin{verbatim}
$ which konqueror
/home/edulix/proyectos/kde4/bin/konqueror 
\end{verbatim}

\section{Compilación del proyecto}

Ahora proseguiremos aplicando el parche a khtml y recompilando khtml con el parche:

\begin{verbatim}
git clone http://github.com/edulix/pfc.git # descargamos el código del presente proyecto
cd kdepim/libkleo
patch -p0 < ../../pfc/find_kleo.patch # parcheamos libkleo
make install # recompilamos y reinstalamos libkleo

cd ../../kdelibs/khtml
patch -p0 < ../../pfc/khtml.patch # parcheamos khtml
make install # recompilamos y reinstalamos khtml
\end{verbatim}

Ha resultado necesario recompilar antes libkleo para que pueda accederse a dicha librería externamente, en nuestro caso desde khtml. Luego hemos recompilado y reinstalado khtml con el parche que implementa el presente proyecto aplicado. Y con esto ya hemos terminado: tenemos un sistema funcional en el que podemos utilizar la nueva extensión HTML descrita en esta memoria en aplicaciones web seguras.

El navegador que utilizaremos para usar la extensión de khtml es Konqueror. De aquí en adelante siempre que vayamos a ejecutar konqueror, previamente arrancaremos un terminal y configuraremos el entorno si no lo hemos hecho antes en ese terminal. Ejemplo:

\begin{verbatim}
$ . ~/proyectos/kde4/environment.sh
$ which konqueror # comprobamos que el binario konqueror es el correcto
/home/edulix/proyectos/kde4/bin/konqueror 
$ konqueror
\end{verbatim}

El mismo comando (\verb|. ~/proyectos/kde4/environment.sh|) nos servirá para poder ejecutar cualquier aplicación de KDE que hayamos compilado, como por ejemplo kleopatra.

\section{Creación de claves GPG}

Evidentemente necesaremos tener un par de claves pública y privada para poder utilizar la extensión. Podemos usar la aplicación kleopatra para este fin. Adjuntamos una captura de pantalla del diálogo con los pasos a seguir para crear una nueva clave GPG con dicha aplicación.

\figura{1}{img/kleopatra-new-key-dialog}{Diálogo para crear una nueva clave}{kleodialog}{}

El primer paso es abrir la aplicación Kleopatra y luego pulsar Ctrl+N para abrir el diálogo de crear un nuevo certificado \ref{kleodialog}. Elegit la opción OpenPGP, escribir nuestro nombre e email, pulsar siguiente, y crear nuestra clave.

\section{Aplicación de prueba: Sweetter}

La aplicación de prueba que finalmente desarrollamos fue un plugin para el software de microblogging Sweetter. Para instalar sweetter necesitamos instalar primero sus dependencias:

\begin{verbatim}
sudo pacman -S django python-pysqlite
\end{verbatim} 

Para descargarnos la última versión de django del repositorio de gitorious donde se alberga, ejecutamos el siguiente comando en el directorio donde vayamos a descargarlo:

\begin{verbatim}
git clone git://gitorious.org/sweetter/sweetter.git
\end{verbatim}

Esto nos creará un directorio ``sweetter/'' donde se descargó la aplicación. Ahora vamos a instalarlo. Al crear la base de datos nos pedirá cierta información como si queremos crear un usuario, le diremos que sí, y daremos los datos del usuario que queramos crear. Esto lo haremos ejecutando el comando \verb|python manage.py syncdb|:

\begin{verbatim}
$ python manage.py syncdb
Creating table auth_permission
Creating table auth_group
Creating table auth_user
Creating table auth_message
Creating table django_admin_log
Creating table django_content_type
Creating table django_session
Creating table django_site
Creating table django_flatpage
Creating table ublogging_profile
Creating table ublogging_option
Creating table ublogging_post
Creating table groups_group
Creating table recoverpw_recover
Creating table karma_karma
Creating table karma_karmasweet
Creating table karma_vote
Creating table karma_log
Creating table followers_follower
Creating table privatetimeline_privatesweet
Creating table jabberbot_jabber

You just installed Django's auth system, which means you don't have any superusers defined.
Would you like to create one now? (yes/no): yes
Username (Leave blank to use 'edulix'): edulix
E-mail address: edulix@gmail.com
Password: 
Password (again): 
Superuser created successfully.
Installing index for auth.Permission model
Installing index for auth.Message model
Installing index for admin.LogEntry model
Installing index for flatpages.FlatPage model
Installing index for ublogging.Option model
Installing index for ublogging.Post model
Installing index for karma.KarmaSweet model
Installing index for karma.Vote model
Installing index for karma.Log model
Installing index for followers.Follower model
Installing index for privatetimeline.PrivateSweet model
Installing index for jabberbot.Jabber model
\end{verbatim}

Ya tenemos Sweetter instalado. Ahora podemos ejecutarlo, de la siguiente manera:

\begin{verbatim}
$ python manage.py runserver
Validating models...
0 errors found

Django version 1.1.1, using settings 'sweetter.settings'
Development server is running at http://127.0.0.1:8000/
Quit the server with CONTROL-C.
\end{verbatim}

Como podemos observar, ya tenemos el servidor sweetter ejecutándose. Si entramos en el navegador en http://127.0.0.1:8000/ veremos la página de inicio de sweetter, y nos permite autenticarnos. Para ello usaremos el usuario y contraseña que elegimos anteriormente. Haciendo clic en la barra de urls superior en ``Profile'' podemos configurar cual es el identificador de nuestra clave GPG, que utilizaremos para recibir mensajes privados cifrados.

Lamentablemente kleopatra no nos muestra para las claves GPG su identificador corto, pero verlo ejecutando el siguiente comando:

\begin{verbatim}
$ gpg --list-keys
pub   1024D/7198F146 2010-06-15 [expires: 2010-07-14]
uid                  Eduardo Robles Elvira <edulix@gmail.com>
sub   1024g/0F8F7B8D 2010-06-15 
\end{verbatim}

\figura{1}{img/sweetter-configure-gpg-key}{Configurando la clave privada del usuario desde su perfil}{sweetter-configure-gpg-key}{}

En nuestro caso el identificar sería 7198F146. Lo introducimos en la página de ``Profile'' de sweetter en el campo correspondiente, como vemos en la figura \ref{sweetter-configure-gpg-key}. 

Ahora podemos ya recibir mensajes cifrados. No hace falta crear otro usuario para que nos envíe un mensaje privado, porque podemos enviarnos un mensaje privado a nosotros mismos. Podemos por ejemplo hacer clic en la barra lateral donde pone ``Send private message'', y el recuadro donde se suele escribir el sweet se convierte en un recuadro donde podemos escribir el mensaje que nos queremos enviar. Luego, podemos ver el mensaje cifrado haciendo clic en la barra de urls superior en la opción ``Private timeline''. El resultado puede verse en \ref{private-timeline-example}.

\figura{1}{img/private-timeline-example}{El mensaje cifrado que previamente nos habíamos enviado aparece en el Private timeline}{private-timeline-example}{}

Para poder enviar un mensaje cifrado a cualquier otro usuario, debemos de tener su clave pública instalada en el sistema. Desde Kleopatra podemos importar claves públicas desde el menu ``File'', tanto desde un fichero como desde un servidor, únicamente es necesario conocer algún dato como el nombre y apellidos o la dirección de correo electrónico de aquella persona cuya clave querramos obtener.
