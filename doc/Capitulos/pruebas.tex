\chapter{Pruebas y conclusiones}\label{pruebas_y_conclusiones}

\section{Pruebas}
\subsection{Aplicación de ejemplo 1: chat web cifrado}

La extensión de HTML que se ha desarrollado tiene como finalidad ser usada en aplicaciones web. Por ello, se ha desarrollado una aplicación de ejemplo que demuestra su uso y a la par sirve como aplicación con la que realizar las pruebas. La primera aplicación de ejemplo que comencé a desarrollar se trataba de un chat web con la tecnología django con python, basándome en una aplicación llamada ``django-chat'' ya existente. Se puede a continuación ver el resultado de ese pequeño desarrollo. No obstante debido a que el código de ``django-chat'' dejaba bastante que desear y también a que surgió una idea más prometedora, opté por finalmente seguir otro camino.

\figura{1}{img/webcypherchat}{Chat cifrado con khtml en el navegador web Konqueror, y el mismo chat motrado en Firefox}{webcypherchat}{}

Como podemos ver en la imagen \ref{webcypherchat}, en la zona de la izquierda se muestra la aplicación de chat tal como se ve en KHTML con la extensión de HTML desarrollada. Los mensajes se envían cifrados, y la lista de mensajes remarca en un recuadro verde los mensajes cifrados recibidos. Haciendo clic derecho en dichos mensajes eligiendo la opción del menú contextual llamada ``Ver detalles del mensaje cifrado...'' se puede comprobar que efectivamente es un mensaje cifrado.

\figura{1}{img/khtml-asking-password}{Diálogo pidiendo al usuario su frase de paso}{khtml-asking-password}{}

Cuando un de claves pública y privada tiene su clave privada cifrada por una frase de paso, durante el renderizado si libkleo no ha cacheado previamente la frase de paso, el diálogo \ref{khtml-asking-password} que pregunta al usuario por su clave privada es mostrado. Una vez se muestra ese diálogo, la frase de paso no se vuelve a mostrar durante un tiempo debido a que libkleo, la librería que utilizamos para cifrar y descifrar, la cachea en memoria. Esto permite descifrar varios mensajes cifrados seguidos sin interrumpir repetidamente al usuario.

\subsection{Aplicación de ejemplo 2: Plugin de mensajes privados en Sweetter}

Sweetter es una aplicación web de microblogging, de hecho probablemente la primera aplicación web software libre de esas características. Fue ideada en un principio por Daniel García Moreno y por el autor de este proyecto, en una primera versión. Su versión 3.0 comenzó su desarrollo como proyecto de fin de carrera de Luis Manuel García Conde y Christian López Espínola, proyecto realizado en este mismo departamento de la Universidad de Sevilla, y su desarrollo continúa. Sweetter 3.0 tiene una arquitectura extensible basada en plugins que pueden ``engancharse'' en varios puntos de extensión. La idea que surgió fue desarrollar un plugin de sweetter que permitiese enviarse mensajes privados cifrados entre los usuarios.

\figura{1}{img/sweetter-private-msg-konqi}{Sweetter enviando un mensaje privado cifrado. También se observa el enlace para enviar mensajes privados al usuario cuyo perfil se está observando en la barra de cabecera y en la barra lateral, así como la imagen con forma de escudo verde que aparece en todos los sweets que sirve para responder al sweet con un mensaje privado}{sweetter-private-msg-konqi}{}

Para hacerlo, se ha añadido en la barra lateral (sidebar) un enlace para que el usuario pueda enviar mensajes privados al usuario que actualmente esté siendo mostrado en dicho sidebar. Además, si el usuario está viendo el perfil de otro usuario, podrá acceder también a dicho enlace en la ``headbar'' o barra de cabecera. En la lista de enlaces superior donde salen las vistas/acciones principales de sweetter, se añadió la vista ``Private timeline'' donde cada usuario puede acceder a sus mensajes privados, que salen en forma de una lista mensajes parecidos a una lista de ``sweets'' corriente.

\figura{1}{img/sweetter-private-timeline}{Un usuario viendo su lista de mensajes privados recibidos (``Private timeline''), y comprobando que un mensaje está cifrado con el menú contextual. Uno de los mensajes privados recibidos no tiene recuadro verde indicando que no está cifrado, debido a que fue enviado mediante Firefox.}{sweetter-private-timeline}{}

Todos los puntos mencionados son puntos de extensión realizados en forma de funciones en el plugin que denominé ``private timeline plugin''. De hecho tuve que añadir un punto de extensión que antes no existía, el de la barra de enlaces, para poder añadir el enlace a la ``Private timeline'' o lista de mensajes. Otro punto de extensión en el que se ``engancha'' el plugin de mensajes privados es el de opciones, donde cada usuario puede configurar la clave privada con la que el resto puede enviarle mensajes cifrados. Y otro más aun, es el de un ``reply privado'', respondiendo a un sweet (mensaje) concreto de un usuario mediante un mensaje privado.

La arquitectura de Sweetter es bastante flexible y ha permitido que el plugin desarrollado  se meta en todos los recovecos de sweetter y se fusione de manera que ahora parece una funcionalidad persistente en todo el flujo de trabajo del usuario. Desarrollar un plugin para sweetter además ha sido bastante agradable y sencillo de hacer debido a que está basado en Django y Python, herramientas de desarrollo web bastante potentes.

La extensión desarrollada ya está en visos de entrar en la web oficial de sweetter, y Daniel García Moreno, mantenedor actual y principal desarrollador, ya ha realizado algunos parches, debido a que el desarrollo se ha realizado mediante un modelo abierto en un repositorio Git. El plugin además permite igualmente enviar y recibir mensajes privados aunque no estén cifrados si se hace desde otro navegador o si no se configura la clave privada del usuario que va a recibir el mensaje, como puede verse en \ref{sweetter-private-timeline}.  

Los mensajes se envían mediante peticiones AJAX, se accede por Javascript al contenido del cuadro de texto ``seguro'',  el cual devuelve no el texto en claro sino el texto cifrado. Además cada vez que se escribe un mensaje nuevo hay que borrar el contenido del cuadro de texto, y cada vez que se escribe un mensaje a un destinatario distinto hay que cambiar la clave con la que se cifra el texto en claro. Nuestra extensión no permite realizar dichas acciones sobre un cuadro de entrada texto (input type=text) seguro porque ello comprometería la privacidad del mensaje, por lo que el cuadro de texto es eliminado y se crea uno nuevo mediante Javascript en esos casos.

\section{Conclusiones}\label{conclusiones} 

Finalmente y como colofón a esta memoria, describiremos en este apartado las distintas conclusiones que hemos obtenido durante la elaboración de este proyecto. Debemos recordar que en dicho proyecto el autor ha desarrollado una extensión de HTML que añade soporte de seguridad ``end to end'' directamente al navegador web con el fin de demostrar el peligro del ``Server in the middle'' y de mejorar mejorar la seguridad de las aplicaciones web para que puedan ser tan seguras como las aplicaciones análogas de escritorio. Además, se ha desarrollado un plugin para la aplicación web Sweetter que muestra el uso potencial que puede tener dicha extensión.

Abordar un proyecto existente de la envergadura de khtml no es sencillo. No obstante el desarrollo de la extensión ha sido exitoso, consiguiendo en efecto desarrollar el plugin propuesto, y cumpliendo con los requisitos seguridad del texto cifrado. 

Uno de los principales objetivos de realizar esta extensión es conseguir simplificar al máximo el uso y aplicación de la seguridad punto a punto en páginas web. A parte de añadir unos pocos atributos HTML y añadir una opción en la configuración del usuario para que indique su clave pública, la extensión de mensajes privados para sweetter no ha requerido ningún otro un esfuerzo adicional para que incluyese soporte de cifrado punto a punto. Además, el mismo plugin puede usarse tal cual en navegadores web que no soporten la extensión sin tener que hacer ninguna modificación de la misma, por lo que es cien por cien reutilizable en ese aspecto.

Este último punto es importante porque permite una migración gradual y sin problemas por parte de usuarios con navegadores que no soporten la extensión. El autor considera importante recalcar la visión global de la extensión que se describe en la presente memoria. En el mundo de la informática como en muchos otros, cuando surge una nueva idea como es en nuestro caso el aplicar la seguridad punto a punto directamente desde el navegador, luego es fácil ver que en realidad es el siguiente paso lógico a dar en la evolución de dicha tecnología, y que tarde o temprano iba a realizarse. Pero es necesario dar el paso para descubrir realmente si se erra en dicho análisis o no.

Por supuesto, la extensión desarrollada sólo es un pequeño paso. Tiene sus limitaciones, que fueron reflejadas en capítulos anteriores, pero sirve como prueba de concepto sólida, basada en el código de una librería de renderizado potente como es KHTML, y demostrando su potencial mediante una la aplicación web de calibre como es Sweetter. 

¿El futuro? El futuro del cifrado punto a punto en el navegador web puede que siga por los caminos expuestos en la primera solución propuesta en el capítulo de Soluciones propuestas, solución en la que se creaba una ``jaula'', un entorno controlado donde es posible acceder y manipular los contenidos seguros para poder crear aplicaciones más ricas y versátiles. Es una apuesta más arriesgada que la que el autor hizo en el presente proyecto, pero también una apuesta con un potencial si cabe más prometedor. Un factor muy importante para conseguir la concienciación de los internautas ante el problema del ``Server in the middle'' y de su solución mediante el cifrado punto a punto pasa por la estandarización de jure de las extensiones que se desarrollen, en el W3C, y de uso de métodos de cifrado y descifrado conocidos y estándares.
