% 2. Introducción, conceptos. arquitectura cliente-servidor, man-in-the-middle, server-in-the-middle, https
\chapter{Introducción}\label{introduccion}
\section{Definiciones y abreviaturas}\label{definicionesyabreviaturas}

Esto es un cap\'itulo de prueba. Esto es un cap\'itulo de prueba.Esto es un cap\'itulo de prueba.
Esto es un cap\'itulo de prueba. Esto es un cap\'itulo de prueba. Esto es un cap\'itulo de prueba. 
Esto es un cap\'itulo de prueba. Esto es un cap\'itulo de prueba. Esto es un cap\'itulo de prueba. 
Esto es un cap\'itulo de prueba. 

Esto es un cap\'itulo de prueba. Esto es un cap\'itulo de prueba. Esto es un cap\'itulo de prueba. 
Esto es un cap\'itulo de prueba. Esto es un cap\'itulo de prueba. Esto es un cap\'itulo de prueba. 
Esto es un cap\'itulo de prueba. 

Esto es una referncia a la bibliograf\'ia \cite{desousa}.



\codigofuente{TeX}{Macro para insertar un cuadro}{codigo/macrotabla}

\figura{0.5}{img/knuth}{Im\'agen de Knuth}{knuth}{}
